% Copyright (C) 2017 Kevin W. Hamlen
%
% This program is free software; you can redistribute it and/or
% modify it under the terms of the GNU General Public License
% as published by the Free Software Foundation; either version 2
% of the License, or (at your option) any later version.
%
% This program is distributed in the hope that it will be useful,
% but WITHOUT ANY WARRANTY; without even the implied warranty of
% MERCHANTABILITY or FITNESS FOR A PARTICULAR PURPOSE.  See the
% GNU General Public License for more details.
%
% You should have received a copy of the GNU General Public License
% along with this program; if not, write to the Free Software
% Foundation, Inc., 51 Franklin Street, Fifth Floor, Boston,
% MA  02110-1301, USA.
%
% The latest version of this program can be obtained from
% http://songs.sourceforge.net.

\documentclass[letterpaper,oneside,landscape]{article}
\usepackage[T1]{fontenc}
\usepackage[pdfpagemode=UseThumbs]{hyperref}
\usepackage{color}
\usepackage[slides]{songs}
% \includeonlysongs{2}

\setlength{\oddsidemargin}{-0.5in}
\setlength{\evensidemargin}{-0.5in}
\setlength{\textwidth}{10in}
\setlength{\topmargin}{-0.75in}
\setlength{\topskip}{0in}
\setlength{\headheight}{13.6pt}
\setlength{\headsep}{0.5in}
\setlength{\textheight}{6.5in}

% Don't number the verses:
\noversenumbers

% Put each verse and chorus on a separate slide:
\sepverses

% Use 28pt Adobe Helvetica font for the lyrics
\renewcommand{\lyricfont}{%
  \fontfamily{phv}\fontseries{b}\fontsize{28pt}{28pt}\selectfont%
}

% If background colors are supported on this machine, then slides
% will have white lettering on a blue background.
\csname @ifundefined\endcsname{set@page@color}{}{
  \definecolor{SlideBG}{rgb}{0,0,0.43}
  \pagecolor{SlideBG}
  \color{white}
  \definecolor{SongbookShade}{rgb}{0,0,0.2}
}

% Define some headers for each slide to help the projector-operator
% find the correct slide.  We use the fancyhdr package for this.
\IfFileExists{fancyhdr.sty}{
  \usepackage{fancyhdr}
  \usepackage{extramarks}
  \pagestyle{fancy}
  \fancyhf{}
  \lhead{\sffamily\firstleftmark}
  \rhead{\sffamily\firstrightmark}
  \renewcommand{\headrulewidth}{0pt}

  \renewcommand{\songmark}{\markboth{}{\thesongnum}}
  \renewcommand{\versemark}{%
    \ifvnumbered
      \markboth{\thesongnum. \songtitle}{Verse \theversenum}%
    \else
      \markboth{\thesongnum. \songtitle}{}%
    \fi
  }
  \renewcommand{\chorusmark}{\markboth{\thesongnum. \songtitle}{Chorus}}
}{}

\begin{document}

\begin{songs}{}
%%
%% This is file `songs.sty',
%% generated with the docstrip utility.
%%
%% The original source files were:
%%
%% songs.dtx  (with options: `package')
%% 
%% This is a generated file.
%% 
%% Copyright (C) 2017 by Kevin W. Hamlen
%% 
%% This file may be distributed and/or modified under the conditions of
%% the LaTeX Project Public License, either version 1.3a of this license
%% or (at your option) any later version.  The latest version of this
%% license is in:
%% 
%%    http://www.latex-project.org/lppl.txt
%% 
%% and version 1.3a or later is part of all distributions of LaTeX version
%% 2004/10/01 or later.
%% 
\NeedsTeXFormat{LaTeX2e}
\ProvidesPackage{songs}
  [2017/06/05 v3.0 Songs package]
\newif\ifSB@etex
\ifx\eTeXversion\undefined\else
  \ifx\eTeXversion\relax\else
    \SB@etextrue
    \ifx\e@alloc\@undefined
      \IfFileExists{etex.sty}{\RequirePackage{etex}}{}
    \fi
  \fi
\fi
\newif\ifSB@pdf\SB@pdffalse
\ifx\pdfoutput\undefined\else
  \ifx\pdfoutput\relax\else
    \ifnum\pdfoutput<\@ne\else
      \SB@pdftrue
    \fi
  \fi
\fi
\newif\ifSB@preamble
\SB@preambletrue
\newif\ifSB@test
\newif\ifSB@testii
\newcommand\SB@temp{}
\newcommand\SB@tempii{}
\newcommand\SB@tempiii{}
\newcommand\SB@tempiv{}
\newcommand\SB@tempv{}
\newcommand\SB@newcount[1]{\@ifdefinable#1{\newcount#1}}
\newcommand\SB@newdimen[1]{\@ifdefinable#1{\newdimen#1}}
\newcommand\SB@newbox[1]{\@ifdefinable#1{\newbox#1}}
\newcommand\SB@newtoks[1]{\@ifdefinable#1{\newtoks#1}}
\newcommand\SB@newwrite[1]{\@ifdefinable#1{\newwrite#1}}
\SB@newdimen\SB@dimen
\SB@newdimen\SB@dimenii
\SB@newdimen\SB@dimeniii
\SB@newdimen\SB@dimeniv
\SB@newbox\SB@box
\SB@newbox\SB@boxii
\SB@newbox\SB@boxiii
\SB@newtoks\SB@toks
\SB@newcount\SB@cnt
\SB@newcount\SB@cntii
\newlength\SB@skip
\SB@newbox\SB@envbox
\RequirePackage{keyval}
\newcommand\SB@app[3]{%
  \expandafter#1\expandafter#2\expandafter{#2#3}%
}
\newcommand\lyricfont{\normalfont\normalsize}
\newcommand\stitlefont{%
  \sffamily\ifslides\Huge\else\slshape\Large\fi%
}
\newcommand\versefont{}
\newcommand\chorusfont{}
\newcommand\notefont{}
\newcommand\meterfont{\tiny\sffamily\upshape}
\newcommand\echofont{%
  \ifdim\fontdimen\@ne\font>\z@\upshape\else\slshape\fi%
}
\newcommand\scripturefont{%
  \usefont{OT1}{pzc}{mb}{it}%
  \shiftdblquotes{-1.1\p@}\z@{-2\p@}\z@%
}
\newcommand\printscrcite[1]{\sffamily\small#1}
\newcommand\snumbgcolor{SongbookShade}
\newcommand\notebgcolor{SongbookShade}
\newcommand\idxbgcolor{SongbookShade}
\newcommand\versejustify{\justifyleft}
\newcommand\chorusjustify{\justifyleft}
\newcommand\notejustify{%
  \advance\baselineskip\p@\relax%
  \leftskip\z@skip\rightskip\z@skip%
  \parfillskip\@flushglue\parindent\z@%
}
\newcommand\placenote[1]{%
  \leftskip\z@skip\rightskip\@flushglue\SB@cbarshift%
  \noindent#1\par%
}
\newcounter{songnum}
\newcounter{versenum}
\renewcommand\thesongnum{\songnumstyle{songnum}}
\newcommand\songnumstyle{}
\let\songnumstyle\arabic
\renewcommand\theversenum{\versenumstyle{versenum}}
\newcommand\versenumstyle{}
\let\versenumstyle\arabic
\newcommand\printsongnum[1]{\sffamily\bfseries\LARGE#1}
\newcommand\printversenum[1]{\lyricfont#1.\ }
\newcommand\placeversenum[1]{%
  \hskip-\leftskip\hskip-\parindent\relax%
  \box#1%
}
\newcommand\everyverse{}
\newcommand\everychorus{}
\newcommand\printchord[1]{\sffamily\slshape\large#1}
\newcommand\chordlocals{}
\newlength\versesep
\versesep123456789sp\relax
\newlength\afterpreludeskip
\afterpreludeskip=2\p@\@plus4\p@
\newlength\beforepostludeskip
\beforepostludeskip=2\p@\@plus4\p@
\newlength\baselineadj
\baselineadj\z@skip
\newcommand\clineparams{%
  \baselineskip\f@size\p@%
  \advance\baselineskip-2\p@%
  \lineskiplimit-2\p@%
  \lineskip-2\p@%
}
\parindent.25in
\newlength\idxheadwidth
\setlength\idxheadwidth{1.5cm}
\newlength\songnumwidth
\settowidth\songnumwidth{\printsongnum{999.}}
\newlength\versenumwidth
\settowidth\versenumwidth{\printversenum{9\kern1em}}
\newlength\cbarwidth
\setlength\cbarwidth\p@
\newlength\sbarheight
\setlength\sbarheight\p@
\interlinepenalty\@m
\SB@newcount\vvpenalty\vvpenalty200
\SB@newcount\ccpenalty\ccpenalty200
\SB@newcount\vcpenalty\vcpenalty200
\SB@newcount\cvpenalty\cvpenalty200
\SB@newcount\brkpenalty\brkpenalty200
\SB@newcount\spenalty\spenalty100
\newcommand\songmark{}
\newcommand\versemark{}
\newcommand\chorusmark{}
\newcommand\extendprelude{\showrefs\showauthors}
\newcommand\extendpostlude{\songcopyright\ \songlicense\unskip}
\newcommand\idxheadfont{\sffamily\bfseries\LARGE}
\newcommand\idxtitlefont{\sffamily\slshape}
\newcommand\idxlyricfont{\rmfamily}
\newcommand\idxscripfont{\sffamily\small\slshape}
\newcommand\idxauthfont{\small\bfseries}
\newcommand\idxrefsfont{\normalfont\normalsize}
\newcommand\idxbook[1]{\small\bfseries#1}
\newcommand\idxcont[1]{\small\textbf{#1} (continued)}
\newcommand\colbotglue{}
\let\colbotglue\z@skip
\newcommand\lastcolglue{}
\let\lastcolglue\@flushglue
\SB@newcount\minfrets\minfrets4
\SB@newdimen\SB@colwidth
\DeclareOption{slides}{\slides}
\newcommand\slides{%
  \slidestrue%
  \def\lyricfont{\normalfont\huge}%
  \def\chorusfont{\slshape}%
  \def\versejustify{\justifycenter}%
  \let\chorusjustify\versejustify
  \def\placenote##1{\justifycenter\noindent##1\par}%
  \scriptureoff%
  \onesongcolumn%
  \ifSB@preamble\ifSB@chordedspec\else\SB@chordsoff\fi\fi%
  \spenalty-\@M%
  \let\colbotglue\@flushglue%
  \setlength\cbarwidth\z@%
  \setlength\sbarheight\z@%
}
\newcommand\justifyleft{%
  \leftskip\parindent%
  \ifSB@inverse\advance\leftskip\versenumwidth\fi%
  \SB@cbarshift%
  \parindent-\parindent%
}
\newcommand\justifycenter{%
  \centering\SB@cbarshift\rightskip\leftskip%
  \def\placeversenum##1{%
    \hskip-\leftskip\hskip-\parindent\relax%
    \hangindent-\wd##1\hangafter\m@ne%
    \box##1\hfil%
  }%
}
\newcommand\SB@outer{\outer}
\DeclareOption{unouter}{\let\SB@outer\relax}
\DeclareOption{rawtext}{\rawtexttrue\indexesoff}
\DeclareOption{noshading}{\SB@colorboxesfalse}
\DeclareOption{noindexes}{\indexesoff}
\newcommand\indexeson{\songindexestrue}
\newcommand\indexesoff{\songindexesfalse}
\DeclareOption{nopdfindex}{%
  \let\songtarget\@gobbletwo%
  \let\songlink\@secondoftwo%
}
\newif\ifSB@measurespec
\newif\ifSB@chordedspec
\DeclareOption{chorded}{\chordson}
\DeclareOption{lyric}{\chordsoff}
\newcommand\chordson{\SB@chordedspectrue\SB@chordson}
\newcommand\chordsoff{\SB@chordedspectrue\SB@chordsoff}
\newcommand\SB@chordson{%
  \ifrawtext%
    \SB@errrtopt%
  \else%
    \chordedtrue\lyricfalse%
    \let\SB@bracket\SB@chord%
    \let\SB@rechord\SB@@rechord%
    \let\SB@ch\SB@ch@on%
    \ifSB@measurespec%
      \ifmeasures\SB@measureson\else\SB@measuresoff\fi%
    \else%
      \SB@measureson%
    \fi%
    \ifSB@preamble\def\colbotglue{\z@\@plus.5\textheight}\fi%
    \SB@setbaselineskip%
  \fi%
}
\newcommand\SB@chordsoff{%
  \chordedfalse\lyrictrue%
  \def\SB@bracket##1]{\ignorespaces}%
  \let\SB@rechord\relax%
  \let\SB@ch\SB@ch@off%
  \ifSB@measurespec%
    \ifmeasures\SB@measureson\else\SB@measuresoff\fi%
  \else%
    \SB@measuresoff%
  \fi%
  \ifSB@preamble\let\colbotglue\z@skip\fi%
  \SB@setbaselineskip%
}
\DeclareOption{showmeasures}{\measureson}
\DeclareOption{nomeasures}{\measuresoff}
\newcommand\measureson{\SB@measurespectrue\SB@measureson}
\newcommand\measuresoff{\SB@measurespectrue\SB@measuresoff}
\newcommand\SB@measureson{%
  \measurestrue%
  \let\SB@mbar\SB@makembar%
  \ifchorded%
    \let\SB@mch\SB@mch@on%
  \else%
    \let\SB@mch\SB@mch@m%
  \fi%
  \ifSB@inverse\SB@loadactives\fi%
  \ifSB@inchorus\SB@loadactives\fi%
}
\newcommand\SB@measuresoff{%
  \measuresfalse%
  \let\SB@mbar\@gobbletwo%
  \ifchorded%
    \let\SB@mch\SB@ch@on%
  \else%
    \let\SB@mch\SB@ch@off%
  \fi%
  \ifSB@inverse\SB@loadactives\fi%
  \ifSB@inchorus\SB@loadactives\fi%
}
\DeclareOption{transposecapos}{\transcapostrue}
\DeclareOption{noscripture}{\SB@omitscriptrue}
\newcommand\scriptureon{\SB@omitscripfalse}
\newcommand\scriptureoff{\SB@omitscriptrue}
\DeclareOption{twosongcolumns}{\SB@numcols\tw@}
\DeclareOption{onesongcolumn}{\SB@numcols\@ne}
\newcommand\songcolumns[1]{%
  \SB@cnt#1\relax%
  \ifnum\SB@cnt=\SB@numcols\else%
    \ifSB@preamble\else{\SB@clearpage}\fi%
  \fi%
  \SB@numcols\SB@cnt%
  \ifnum\SB@numcols>\z@%
    \SB@colwidth-\columnsep%
    \multiply\SB@colwidth\SB@numcols%
    \advance\SB@colwidth\columnsep%
    \advance\SB@colwidth\textwidth%
    \divide\SB@colwidth\SB@numcols%
  \else%
    \ifrepchorus\SB@warnrc\fi%
  \fi%
}
\newcommand\onesongcolumn{\songcolumns\@ne}
\newcommand\twosongcolumns{\songcolumns\tw@}
\newcommand\songlist{}
\newcommand\includeonlysongs[1]{%
  \ifSB@songsenv\SB@errpl\else%
    \partiallisttrue%
    \renewcommand\songlist{#1}%
  \fi%
}
\newcommand\nosongnumbers{\setlength\songnumwidth\z@}
\newcommand\noversenumbers{%
  \renewcommand\printversenum[1]{}%
  \setlength\versenumwidth\z@%
}
\ifSB@etex
  \newcommand\repchoruses{%
    \ifnum\SB@numcols<\@ne\SB@warnrc\fi%
    \repchorustrue%
  }
\else
  \newcommand\repchoruses{\SB@erretex}
\fi
\newcommand\norepchoruses{\repchorusfalse}
\newcommand\sepverses{%
  \vvpenalty-\@M%
  \ccpenalty100 %
  \vcpenalty\vvpenalty%
  \cvpenalty\vvpenalty%
  \let\colbotglue\@flushglue%
}
\AtBeginDocument{
  \SB@setversesep
  \songcolumns\SB@numcols
  \SB@preamblefalse
}
\SB@newbox\SB@songbox
\SB@newcount\SB@numcols\SB@numcols\tw@
\SB@newcount\SB@colnum
\SB@newbox\SB@colbox
\SB@newbox\SB@pgbox
\SB@newbox\SB@mrkbox
\newcommand\SB@maxmin[3]{\ifdim#1#2#3#1#3\fi}
\newcommand\SB@mkpage[4]{%
  \ifvoid#2\else\begingroup%
    \edef\SB@temp{\ifnum#2=\SB@box\SB@boxii\else\SB@box\fi}%
    \edef\SB@tempii{\ifnum#2=\SB@boxiii\SB@boxii\else\SB@boxiii\fi}%
    \splitmaxdepth\maxdepth\splittopskip\z@skip%
    \ifnum#1=\z@\global\setbox#2\vbox{\unvbox#2\vfil}\fi%
    \loop\ifnum#3<\SB@numcols%
      \ifnum#1=\z@\setbox\SB@tempii\copy#2\fi%
      \setbox\SB@temp\vsplit#2to#4\relax%
      \ifvoid#2%
        \ifnum#1=\z@%
          \global\setbox#2\box\SB@tempii%
        \else%
          \SB@updatepage%
          \global\advance#3\@ne%
        \fi%
        #3\SB@numcols%
      \else%
        \SB@updatepage%
        \global\advance#3\@ne%
        \ifrepchorus\ifvoid\SB@chorusbox\else%
          \SB@insertchorus#2%
        \fi\fi%
      \fi%
    \repeat%
    \ifnum#1=\z@\global\setbox#2\vbox{\unvbox#2\unskip}\fi%
  \endgroup\fi%
}
\newcommand\SB@migrate[1]{%
  \SB@toks\expandafter{#1}%
  \edef\SB@temp{\the\SB@toks}%
  \ifx\SB@temp\@empty\else\mark{\the\SB@toks}\fi%
}
\newcommand\SB@updatepage{%
  \global\setbox\SB@mrkbox\vbox{%
    \unvbox\SB@mrkbox%
    \SB@migrate\splitfirstmark%
    \SB@migrate\splitbotmark%
  }%
  \global\setbox\SB@pgbox\hbox{%
    \SB@dimen\SB@colwidth%
    \advance\SB@dimen\columnsep%
    \multiply\SB@dimen\SB@colnum%
    \advance\SB@dimen-\wd\SB@pgbox%
    \unhbox\SB@pgbox%
    \ifdim\SB@dimen=\z@\else\hskip\SB@dimen\relax\fi%
    \box\SB@temp%
  }%
}
\newcommand\SB@droppage{\setbox\SB@temp\box\voidb@x}
\newcommand\SB@output[1]{%
  \ifnum\SB@numcols>\z@\begingroup%
    \loop%
      \SB@dimen\textheight%
      \ifinner\else\advance\SB@dimen-\pagetotal\fi%
      \SB@mkpage#1\SB@colbox\SB@colnum\SB@dimen%
      \SB@testfalse\SB@testiitrue%
      \ifnum#1>\@ne\ifvoid\SB@colbox\ifnum\SB@colnum>\z@%
        \SB@testtrue\SB@testiifalse%
      \fi\fi\fi%
      \ifnum\SB@colnum<\SB@numcols\SB@testiifalse\else\SB@testtrue\fi%
      \ifSB@test%
        \unvbox\SB@mrkbox%
        \ifinner\else\kern\z@\fi%
        \box\SB@pgbox%
        \ifinner\else\vfil\break\vskip\vsize\relax\fi%
        \global\SB@colnum\z@%
      \fi%
    \ifSB@testii\repeat%
  \endgroup\else%
    \unvbox\SB@colbox\unskip%
  \fi%
}
\newcommand\SB@putboxes[1]{%
  \SB@dimen\ifnum\SB@numcols>\z@\ht\SB@colbox\else\p@\fi%
  #1\SB@colbox%
  \ifdim\SB@dimen>\z@%
    \SB@breakpoint\spenalty%
    \ifdim\sbarheight>\z@%
      \vskip-\sbarheight\relax%
    \fi%
  \fi%
  #1\SB@songbox%
}
\newcommand\SB@nextcol[2]{%
  \ifnum#1>\z@%
    \ifnum\SB@numcols>\z@%
      \global\setbox\SB@colbox\vbox{%
        \SB@cnt#1\relax%
        \SB@dimen\ht\SB@colbox%
        \unvbox\SB@colbox%
        \unskip%
        \ifdim\SB@dimen>\z@%
          \vskip#2\relax%
          \break%
          \advance\SB@cnt\m@ne%
        \fi%
        \loop\ifnum\SB@cnt>\z@%
          \nointerlineskip%
          \null%
          \vfil%
          \break%
          \advance\SB@cnt\m@ne%
        \repeat%
      }%
      \SB@output1%
    \else%
      \ifnum\lastpenalty=-\@M\null\fi%
      \break%
    \fi%
  \fi%
}
\newcommand\SB@selectcol{}
\newcommand\SB@@selectcol{%
  \begingroup%
    \SB@cnt\z@%
    \vbadness\@M\vfuzz\maxdimen%
    \let\SB@updatepage\SB@droppage%
    \SB@dimen\textheight%
    \ifinner\else\advance\SB@dimen-\pagetotal\fi%
    \setbox\SB@boxii\vbox{\SB@putboxes\unvcopy}%
    \SB@cntii\SB@colnum%
    \SB@mkpage0\SB@boxii\SB@cntii\SB@dimen%
    \SB@spos%
    \global\SB@cnt\SB@cnt%
  \endgroup%
  \SB@nextcol\SB@cnt\colbotglue%
}
\newcommand\SB@spbegnew{%
  \setbox\SB@boxiii\copy\SB@songbox%
  \SB@cntii\z@%
  \SB@mkpage0\SB@boxiii\SB@cntii\textheight%
}
\newcommand\SB@spextold{%
  \ifodd\c@page\else%
    \SB@cntii\z@%
    \SB@mkpage0\SB@boxii\SB@cntii\textheight%
  \fi%
}
\newcommand\SB@spextnew{%
  \SB@cntii\z@%
  \SB@mkpage0\SB@boxiii\SB@cntii\textheight%
}
\newcommand\SB@spdblpg{%
  \ifnum\SB@cntii<\SB@numcols%
    \SB@cnt\SB@numcols%
    \advance\SB@cnt-\SB@colnum%
    \if@twoside\ifodd\c@page\else%
      \advance\SB@cnt\SB@numcols%
    \fi\fi%
  \fi%
}
\newcommand\SB@sposi{%
  \ifnum\SB@cntii<\SB@numcols\else\if@twoside%
    \SB@spextold%
  \fi\fi%
  \ifnum\SB@cntii<\SB@numcols\else%
    \SB@spbegnew%
    \ifnum\SB@cntii<\SB@numcols\else\if@twoside%
      \SB@spextnew%
    \fi\fi%
    \SB@spdblpg%
  \fi%
}
\newcommand\SB@sposii{%
  \ifnum\SB@cntii<\SB@numcols\else%
    \SB@spbegnew%
    \ifnum\SB@cntii<\SB@numcols%
      \SB@cnt\SB@numcols%
      \advance\SB@cnt-\SB@colnum%
    \else%
      \if@twoside%
        \SB@spextold%
        \ifnum\SB@cntii<\SB@numcols\else%
          \SB@spextnew%
          \SB@spdblpg%
        \fi%
      \fi%
    \fi%
  \fi%
}
\newcommand\SB@sposiii{%
  \ifnum\SB@cntii>\SB@colnum%
    \SB@cnt\SB@colnum%
    \advance\SB@cnt\@ne%
    \ifnum\SB@cnt<\SB@numcols%
      \setbox\SB@boxiii\copy\SB@songbox%
      \SB@mkpage0\SB@boxiii\SB@cnt\SB@dimen%
      \advance\SB@cnt\m@ne%
    \fi%
    \ifnum\SB@cnt>\SB@colnum%
      \SB@cnt\z@%
      \SB@sposii%
    \else%
      \SB@cnt\@ne%
    \fi%
  \fi%
}
\newcommand\songpos[1]{%
  \ifcase#1%
    \let\SB@selectcol\relax%
    \let\SB@spos\relax%
  \or%
    \let\SB@selectcol\SB@@selectcol%
    \let\SB@spos\SB@sposi%
  \or%
    \let\SB@selectcol\SB@@selectcol%
    \let\SB@spos\SB@sposii%
  \or%
    \let\SB@selectcol\SB@@selectcol%
    \let\SB@spos\SB@sposiii%
  \else%
    \SB@errspos%
  \fi%
}
\newcommand\SB@spos{}
\songpos\thr@@
\newcommand\SB@clearpage{%
  \SB@testtrue%
  \ifvoid\SB@pgbox%
    \ifdim\ht\SB@colbox=\z@\ifdim\dp\SB@colbox=\z@%
      \SB@testfalse%
    \fi\fi%
  \fi%
  \ifSB@test%
    \SB@cnt\SB@numcols%
    \advance\SB@cnt-\SB@colnum%
    \SB@nextcol\SB@cnt\lastcolglue%
    \SB@output2%
  \fi%
}
\newcommand\SB@cleardpage{%
  \SB@clearpage%
  \if@twoside\ifodd\c@page%
    \SB@nextcol\SB@numcols\@flushglue%
    \SB@output2%
  \fi\fi%
}
\newcommand\SB@stype{\SB@stypcol}
\newcommand\SB@stypcol{%
  \ifnum\SB@numcols>\z@%
    \SB@selectcol%
    \global\setbox\SB@colbox\vbox{\SB@putboxes\unvbox}%
    \SB@output0%
  \else%
    \unvbox\voidb@x%
    \SB@breakpoint\spenalty%
    \ifdim\sbarheight>\z@%
      \vskip-\sbarheight\relax%
    \fi%
    \unvbox\SB@songbox%
  \fi%
}
\newcommand\SB@styppage{%
  \ifnum\SB@numcols>\z@%
    \SB@clearpage%
    \unvbox\SB@songbox%
    \nointerlineskip\null%
  \else%
    \unvbox\SB@songbox%
  \fi%
  \nointerlineskip%
}
\newcommand\SB@sgroup{}
\let\SB@sgroup\@empty
\SB@newcount\SB@groupcnt
\newcommand\SB@clearpboxes{}
\newcommand\SB@partbox[1]{%
  \SB@newbox#1%
  \SB@app\gdef\SB@clearpboxes{\setbox#1\box\voidb@x}%
  \global\setbox#1\box%
}
\newcommand\SB@submitpart{%
  \ifx\SB@sgroup\@empty\else%
    \SB@testfalse
    \@for\SB@temp:=\songlist\do{\ifx\SB@temp\SB@sgroup\SB@testtrue\fi}%
    \ifSB@test%
      \edef\SB@tempii{\SB@sgroup @\the\SB@groupcnt}%
      \expandafter\SB@partbox
        \csname songbox@\SB@tempii\endcsname\SB@songbox%
      \global\expandafter\let%
        \csname stype@\SB@tempii\endcsname\SB@stype%
      \ifrepchorus\ifvoid\SB@chorusbox\else%
        \expandafter\SB@partbox
  \csname chbox@\SB@tempii\endcsname\SB@chorusbox%
      \fi\fi%
    \fi%
    \global\advance\SB@groupcnt%
      \ifnum\SB@groupcnt<\z@\m@ne\else\@ne\fi%
  \fi%
  \setbox\SB@songbox\box\voidb@x%
  \setbox\SB@chorusbox\box\voidb@x%
}
\newcommand\SB@submitsong{%
  \ifpartiallist\SB@submitpart\else\SB@stype\fi%
}
\newcommand\SB@submitenv{%
  \begingroup%
    \let\SB@songbox\SB@envbox%
    \SB@styppage%
  \endgroup%
}
\newcommand\SB@songlistbrk{}
\def\SB@songlistbrk{brk}
\newcommand\SB@songlistnc{}
\def\SB@songlistnc{nextcol}
\newcommand\SB@songlistcp{}
\def\SB@songlistcp{sclearpage}
\newcommand\SB@songlistcdp{}
\def\SB@songlistcdp{scleardpage}
\newcommand\commitsongs{%
  \ifpartiallist%
    \ifnum\SB@numcols>\z@%
      \@for\SB@temp:=\songlist\do{%
        \ifx\SB@temp\SB@songlistnc\SB@nextcol\@ne\@flushglue\else%
        \ifx\SB@temp\SB@songlistbrk\SB@nextcol\@ne\colbotglue\else%
        \ifx\SB@temp\SB@songlistcp\SB@clearpage\else%
        \ifx\SB@temp\SB@songlistcdp\SB@cleardpage\else%
          \SB@groupcnt\m@ne\SB@finloop%
          \SB@groupcnt\z@\SB@finloop%
        \fi\fi\fi\fi%
      }%
    \else%
      \@for\SB@temp:=\songlist\do{%
        \ifx\SB@temp\SB@songlistnc\vfil\break\else%
        \ifx\SB@temp\SB@songlistbrk\break\else%
        \ifx\SB@temp\SB@songlistcp\clearpage\else%
        \ifx\SB@temp\SB@songlistcdp%
          \clearpage%
          \ifodd\c@page\null\newpage\fi%
        \else%
          \SB@groupcnt\m@ne\SB@finloop%
          \SB@groupcnt\z@\SB@finloop%
        \fi\fi\fi\fi%
      }%
    \fi%
    \SB@clearpboxes%
  \fi%
  \SB@clearpage%
}
\newcommand\SB@finloop{%
  \loop\edef\SB@tempii{\SB@temp @\the\SB@groupcnt}%
       \expandafter\ifx%
         \csname songbox@\SB@tempii\endcsname\relax\else%
    \setbox\SB@songbox\expandafter\copy%
        \csname songbox@\SB@tempii\endcsname%
    \expandafter\ifx\csname chbox@\SB@tempii\endcsname\relax%
      \repchorusfalse%
    \else%
      \repchorustrue%
      \setbox\SB@chorusbox\expandafter\copy%
        \csname chbox@\SB@tempii\endcsname%
    \fi%
    \csname stype@\SB@tempii\endcsname%
    \advance\SB@groupcnt\ifnum\SB@groupcnt<\z@\m@ne\else\@ne\fi%
  \repeat%
}
\newcommand\SB@insertchorus[1]{{%
  \vbadness\@M\vfuzz\maxdimen%
  \setbox\SB@box\copy#1%
  \setbox\SB@box\vsplit\SB@box to\maxdimen%
  \edef\SB@temp{\splitfirstmarks\SB@nocmarkclass}%
  \ifx\SB@temp\SB@nocmark\else%
    \edef\SB@temp{\splitfirstmarks\SB@cmarkclass}%
    \ifx\SB@temp\SB@cmark%
      \SB@dimen4096\p@%
      \SB@dimenii\maxdimen%
      \SB@dimeniii\SB@dimen%
      \loop%
        \SB@dimeniii.5\SB@dimeniii%
        \setbox\SB@box\copy#1%
        \setbox\SB@box\vsplit\SB@box to\SB@dimen%
        \edef\SB@temp{\splitfirstmarks\SB@cmarkclass}%
        \ifx\SB@temp\SB@cmark%
          \SB@dimenii\SB@dimen%
          \advance\SB@dimen-\SB@dimeniii%
        \else%
          \advance\SB@dimen\SB@dimeniii%
        \fi%
      \ifdim\SB@dimeniii>2\p@\repeat%
      \setbox\SB@box\vsplit#1to\SB@dimenii%
      \global\setbox#1\vbox{%
        \unvbox\SB@box\unskip%
        \SB@inversefalse\SB@prevversetrue\SB@stanzabreak%
        \SB@putbox\unvcopy\SB@chorusbox%
        \SB@inversetrue\SB@prevversefalse\SB@stanzabreak%
        \unvbox#1%
      }%
    \else\ifx\SB@temp\SB@lastcmark%
      \global\setbox#1\vbox{%
        \unvbox#1%
        \unskip%
        \ifdim\sbarheight>\z@%
          \setbox\SB@box\lastbox%
          \unskip\unpenalty%
        \fi%
        \setbox\SB@box\lastbox%
        \unskip\unskip%
        \SB@inversefalse\SB@prevversetrue\SB@stanzabreak%
        \marks\SB@nocmarkclass{\SB@nocmark}%
        \unvcopy\SB@chorusbox%
        \vskip\versesep\vskip\beforepostludeskip\relax%
        \nointerlineskip\box\SB@box%
        \ifdim\sbarheight>\z@%
          \nobreak\vskip2\p@\@plus\p@%
          \hrule\@height\sbarheight\@width\SB@colwidth%
        \fi%
      }%
    \fi\fi%
  \fi%
}}
\newcommand\nextcol{%
  \@ifstar{\SB@nextcol\@ne\@flushglue}%
          {\ifpartiallist\else\SB@nextcol\@ne\@flushglue\fi}%
}
\newcommand\sclearpage{%
  \@ifstar\SB@clearpage{\ifpartiallist\else\SB@clearpage\fi}%
}
\newcommand\scleardpage{%
  \@ifstar\SB@cleardpage{\ifpartiallist\else\SB@cleardpage\fi}%
}
\newcommand\SB@lop[1]{\expandafter\SB@@lop\the#1\SB@@lop#1}
\newcommand\SB@@lop{}
\def\SB@@lop\\#1\\#2\SB@@lop#3#4{\global#3{\\#2}\global#4{#1}}
\newcommand\SB@emptylist{}
\def\SB@emptylist{\\}
\newcommand\SB@ifempty[3]{%
  \edef\SB@temp{\the#1}%
  \ifx\SB@temp\SB@emptylist#2\else#3\fi%
}
\SB@newtoks\SB@titlelist
\SB@newtoks\SB@titletail
\newcommand\songtitle{}
\newcommand\resettitles{%
  \global\SB@titletail\SB@titlelist%
  \nexttitle%
}
\newcommand\nexttitle{%
  \SB@ifempty\SB@titletail{%
    \global\let\songtitle\relax%
  }{%
    \SB@lop\SB@titletail\SB@toks%
    \edef\songtitle{\the\SB@toks}%
  }%
}
\newcommand\foreachtitle[1]{%
  \ifx\songtitle\relax\else%
    \loop#1\nexttitle\ifx\songtitle\relax\else\repeat%
  \fi%
}
\newif\ifSB@songsenv\SB@songsenvfalse
\newif\ifSB@insong\SB@insongfalse
\newif\ifSB@intersong\SB@intersongfalse
\newif\ifSB@inverse\SB@inversefalse
\newif\ifSB@inchorus\SB@inchorusfalse
\newcommand\SB@closeall{}
\newcommand\SB@rawrefs{}
\newcommand\songauthors{}
\newcommand\songcopyright{}
\newcommand\songlicense{}
\newcommand\songrefs{}
\newcommand\setlicense{\gdef\songlicense}
\newcommand\SB@clearbskeys{}
\newcommand\newsongkey[2]{%
  \SB@app\gdef\SB@clearbskeys{#2}%
  \define@key{beginsong}{#1}%
}
\newsongkey{sr}{\def\SB@rawrefs{}\gdef\songrefs{}}
               {\def\SB@rawrefs{#1}\SB@parsesrefs{#1}}
\newsongkey{by}{\def\songauthors{}}{\def\songauthors{#1}}
\newsongkey{cr}{\def\songcopyright{}}{\def\songcopyright{#1}}
\newsongkey{li}{\setlicense{}}{\setlicense{#1}}
\newsongkey{index}{}{\indexentry{#1}}
\newsongkey{ititle}{}{\indextitleentry{#1}}
\newenvironment{song}{\beginsong}{\SB@endsong}
\newcommand\beginsong[1]{%
  \ifSB@insong\SB@errboo\SB@closeall\fi%
  \ifSB@intersong\SB@errbor\SB@closeall\fi%
  \SB@insongtrue%
  \def\SB@closeall{\endsong}%
  \SB@parsetitles{#1}%
  \global\setbox\SB@songwrites\box\voidb@x%
  \SB@clearbskeys%
  \@ifnextchar[\SB@bskvfmt\SB@@beginsong%
}
\newcommand\SB@@beginsong{%
  \@ifnextchar\bgroup\SB@bsoldfmt\SB@@@beginsong%
}
\newcommand\SB@bsoldfmt[3]{%
  \SB@bskvfmt[sr={#1},by={#2},cr={#3}]%
}
\newcommand\SB@bskvfmt{}
\def\SB@bskvfmt[#1]{%
  \setkeys{beginsong}{#1}%
  \SB@@@beginsong%
}
\newcommand\SB@@@beginsong{%
  \global\SB@stanzafalse%
  \setbox\SB@chorusbox\box\voidb@x%
  \SB@gotchorusfalse%
  \setbox\SB@songbox\vbox\bgroup\begingroup%
    \ifnum\SB@numcols>\z@\hsize\SB@colwidth\fi%
    \leftskip\z@skip\rightskip\z@skip%
    \parfillskip\@flushglue\parskip\z@skip%
    \SB@raggedright%
    \global\SB@transposefactor\z@%
    \global\SB@cr@{\\}%
    \protected@edef\@currentlabel{\p@songnum\thesongnum}%
    \setcounter{versenum}{1}%
    \SB@prevversetrue%
    \meter44%
    \resettitles%
    \SB@addtoindexes\songtitle\SB@rawrefs\songauthors%
    \nexttitle%
    \foreachtitle{\expandafter\SB@addtotitles\expandafter{\songtitle}}%
    \resettitles%
    \lyricfont\relax%
    \SB@setbaselineskip%
}
\newcommand\SB@endsong{%
  \ifSB@insong%
      \ifSB@inverse\SB@erreov\endverse\fi%
      \ifSB@inchorus\SB@erreoc\endchorus\fi%
      \global\SB@skip\versesep%
      \unskip%
      \ifrepchorus\ifvoid\SB@chorusbox\else%
        \ifSB@prevverse\ifvnumbered%
          \marks\SB@cmarkclass{\SB@lastcmark}%
        \fi\fi%
      \fi\fi%
    \endgroup\egroup%
    \begingroup%
      \ifnum\SB@numcols>\z@%
        \hsize\ifpagepreludes\textwidth\else\SB@colwidth\fi%
      \fi%
      \leftskip\z@skip\rightskip\z@skip%
      \parfillskip\@flushglue\parskip\z@skip\parindent\z@%
      \global\setbox\SB@envbox\vbox{%
        \songmark%
        \unvbox\SB@songwrites%
        \ifpagepreludes\else\ifdim\sbarheight>\z@%
          \hrule\@height\sbarheight\@width\hsize%
          \nobreak\vskip5\p@\relax%
        \fi\fi%
        \resettitles%
        \begingroup%
          \songtarget{\ifnum\c@section=\z@1\else2\fi}%
                     {song\theSB@songsnum-\thesongnum}%
        \endgroup%
        \vbox{\makeprelude}%
        \nobreak\vskip\SB@skip%
        \vskip\afterpreludeskip\relax%
      }%
      \ifnum\SB@numcols>\z@\hsize\SB@colwidth\fi%
      \global\setbox\SB@songbox\vbox{%
        \ifpagepreludes\else\unvbox\SB@envbox\fi%
        \unvbox\SB@songbox%
        \nobreak\vskip\SB@skip%
        \vskip\beforepostludeskip\relax%
        \nointerlineskip%
        \vbox{\makepostlude}%
        \ifdim\sbarheight>\z@%
          \nobreak\vskip2\p@\@plus\p@%
          \nointerlineskip%
          \hbox{\vrule\@height\sbarheight\@width\hsize}%
        \fi%
      }%
    \endgroup%
    \SB@insongfalse%
    \edef\SB@sgroup{\thesongnum}%
    \global\SB@groupcnt\z@%
    \ifpagepreludes\SB@submitenv\fi%
    \SB@submitsong%
    \ifnum\SB@grouplvl=\z@\let\SB@sgroup\@empty\fi%
    \stepcounter{songnum}%
  \else%
    \ifSB@intersong\SB@erreor\SB@closeall%
    \else\SB@erreot\fi%
  \fi%
}
\newcommand\SB@setbaselineskip{%
  \SB@dimen\f@size\p@%
  \baselineskip\SB@dimen\relax%
  \ifchorded%
    \setbox\SB@box\hbox{{\printchord{ABCDEFG\shrp\flt/j7}}}%
    \advance\baselineskip\ht\SB@box%
    \advance\baselineskip2\p@%
  \fi%
  \ifslides%
    \advance\baselineskip.2\SB@dimen\@plus.5\SB@dimen%
      \@minus.2\SB@dimen%
  \else%
    \advance\baselineskip\z@\@plus.1\SB@dimen\relax%
  \fi%
  \advance\baselineskip\baselineadj%
}
\newcommand\SB@setversesep{%
  \SB@dimen123456789sp%
  \edef\SB@temp{\the\SB@dimen}%
  \edef\SB@tempii{\the\versesep}%
  \ifx\SB@temp\SB@tempii%
    \begingroup%
      \lyricfont\relax%
      \SB@dimen\f@size\p@%
      \ifchorded%
        \setbox\SB@box\hbox{{\printchord{ABCDEFG\shrp\flt/j7}}}%
        \advance\SB@dimen\ht\SB@box%
      \fi%
      \ifslides%
        \global\versesep1.2\SB@dimen\@plus.3\SB@dimen%
        \@minus.3\SB@dimen%
      \else%
        \global\versesep.75\SB@dimen\@plus.25\SB@dimen%
        \@minus.13\SB@dimen%
      \fi%
    \endgroup%
  \fi%
}
\newcommand\makeprelude{%
  \resettitles%
  \ifslides%
    \hbox to\hsize{{\hfil\stitlefont\relax\songtitle\hfil}}%
    \vskip5\p@%
    \hbox to\hsize{%
      \hfil%
      \vbox{%
        \divide\hsize\tw@\parskip\p@\relax%
        \centering\small\extendprelude%
      }%
      \hfil%
    }%
  \else%
    \ifdim\songnumwidth>\z@%
      \setbox\SB@boxii\hbox{{\SB@colorbox\snumbgcolor{%
        \hbox to\songnumwidth{%
          \printsongnum{\thesongnum}\hfil%
        }%
      }}}%
    \fi%
    \setbox\SB@box\vbox{%
      \ifdim\songnumwidth>\z@%
        \SB@dimen\wd\SB@boxii%
        \advance\SB@dimen3\p@%
        \ifpagepreludes\multiply\SB@dimen\tw@\fi%
        \advance\hsize-\SB@dimen%
      \fi%
      \ifpagepreludes\centering\else\SB@raggedright\fi%
      \offinterlineskip\lineskip\p@%
      {\stitlefont\relax%
       \songtitle\par%
       \nexttitle%
       \foreachtitle{(\songtitle)\par}}%
      \ifdim\prevdepth=\z@\kern\p@\fi%
      \parskip\p@\relax\tiny%
      \extendprelude%
      \kern\z@%
    }%
    \ifdim\songnumwidth>\z@%
      \hbox{%
        \ifdim\ht\SB@boxii>\ht\SB@box%
          \box\SB@boxii%
          \kern3\p@%
          \vtop{\box\SB@box}%
        \else%
          \SB@colorbox\snumbgcolor{\vbox to\ht\SB@box{{%
            \hbox to\songnumwidth{%
              \printsongnum{\thesongnum}\hfil%
            }\vfil%
          }}}%
          \kern3\p@%
          \box\SB@box%
        \fi%
      }%
    \else%
      \unvbox\SB@box%
    \fi%
  \fi%
}
\newcommand\makepostlude{%
  \SB@raggedright\baselineskip\z@skip\parskip\z@skip\parindent\z@%
  \tiny\extendpostlude%
}
\newcommand\showauthors{%
  \setbox\SB@box\hbox{\bfseries\sfcode`.\@m\songauthors}%
  \ifdim\wd\SB@box>\z@\unhbox\SB@box\par\fi%
}
\newcommand\showrefs{%
  \setbox\SB@box\hbox{\slshape\songrefs\vphantom,}%
  \ifdim\wd\SB@box>\z@\unhbox\SB@box\par\fi%
}
\newcommand\SB@next{}
\newcommand\SB@donext{}
\newcommand\SB@dothis{}
\newcommand\SB@nextname{}
\newcommand\SB@appendsp[1]{#1\expandafter{\the#1 }}
\newcommand\SB@parsetitles[1]{%
  \begingroup%
    \global\SB@titlelist{\\}%
    \SB@toks{}%
    \let\\\SB@titlesep%
    \SB@pthead#1\SB@endparse%
  \endgroup%
}
\newcommand\SB@pthead{\futurelet\SB@next\SB@@pthead}
\newcommand\SB@@pthead{%
  \ifcat\noexpand\SB@next\@sptoken%
    \expandafter\SB@@@pthead%
  \else%
    \expandafter\SB@ptmain%
  \fi%
}
\newcommand\SB@@@pthead{%
  \afterassignment\SB@pthead%
  \let\SB@next= }
\newcommand\SB@ptloop{\futurelet\SB@next\SB@ptmain}
\newcommand\SB@ptmain{%
  \ifcat\noexpand\SB@next\@sptoken%
    \let\SB@donext\SB@ptsp%
  \else\ifcat\noexpand\SB@next\bgroup%
    \let\SB@donext\SB@ptbg%
  \else\ifx\SB@next\SB@endparse%
    \global\SB@titlelist\expandafter{\the\SB@titlelist\\}%
    \let\SB@donext\@gobble%
  \else\ifx\SB@next\\%
    \SB@toks{}%
    \def\SB@donext{\SB@ptstep\SB@pthead}%
  \else%
    \def\SB@donext{\SB@ptstep\SB@ptloop}%
  \fi\fi\fi\fi%
  \SB@donext}
\newcommand\SB@ptstep[2]{%
  \global\SB@titlelist\expandafter\expandafter\expandafter{%
    \expandafter\the\expandafter\SB@titlelist\the\SB@toks#2}%
  \SB@toks{}%
  #1}
\newcommand\SB@ptbg[1]{\SB@ptstep\SB@ptloop{{#1}}}
\newcommand\SB@ptsp{
  \SB@appendsp\SB@toks%
  \afterassignment\SB@ptloop%
  \let\SB@next= }
\newcommand\SB@titlesep{SB@titlesep}
\newcommand\SB@endparse{%
  \SB@Error{Title parsing failed}{This error should not occur.}%
}
\newcommand\SB@parsesrefs[1]{%
  \begingroup%
    \SB@toks{\begingroup\SB@sractives}%
    \SB@prloop#1\SB@endparse%
    \xdef\songrefs{\the\SB@toks\endgroup}%
  \endgroup%
}
\newcommand\SB@prloop{\futurelet\SB@next\SB@prstep}
\newcommand\SB@prstep{%
  \ifcat\noexpand\SB@next A%
    \expandafter\SB@prcpy%
  \else%
    \expandafter\SB@@prstep%
  \fi%
}
\newcommand\SB@@prstep{%
  \ifcat\noexpand\SB@next\@sptoken%
    \let\SB@donext\SB@prspace%
  \else\ifx\SB@next-%
    \let\SB@donext\SB@prhyphen%
  \else\ifx\SB@next,%
    \let\SB@donext\SB@prcomma%
  \else\ifx\SB@next\SB@endparse%
    \let\SB@donext\@gobble%
  \else\ifcat\noexpand\SB@next\bgroup%
    \let\SB@donext\SB@prgr%
  \else%
    \let\SB@donext\SB@prcpy%
  \fi\fi\fi\fi\fi%
  \SB@donext%
}
\newcommand\SB@prcpy[1]{\SB@toks\expandafter{\the\SB@toks#1}\SB@prloop}
\newcommand\SB@prgr[1]{\SB@toks\expandafter{\the\SB@toks{#1}}\SB@prloop}

\newcommand\SB@prcomma[1]{}
{\catcode`,\active
 \gdef\SB@prcomma#1{\SB@toks\expandafter{\the\SB@toks,}\SB@prloop}}
\newcommand\SB@prhyphen[1]{}
{\catcode`-\active
 \gdef\SB@prhyphen#1{\SB@toks\expandafter{\the\SB@toks-}\SB@prloop}}
\newcommand\SB@prspace[1]{}
{\obeyspaces
\gdef\SB@prspace{\SB@toks\expandafter{\the\SB@toks }\SB@@prspace}}
\newcommand\SB@@prspace{\afterassignment\SB@prloop\let\SB@temp= }

\newcommand\SB@sractives{}
{\catcode`,\active\catcode`-\active\obeyspaces%
\gdef\SB@sractives{%
\let,\SB@srcomma\let-\SB@srhyphen\let \SB@srspace%
\SB@srspacing}%
}
\newcommand\SB@srspacing{%
  \nonfrenchspacing\sfcode`\;=1500\sfcode`\,=1250\relax%
}
\newcommand\SB@srcomma{,\futurelet\SB@next\SB@@srcomma}
\newcommand\SB@@srcomma{%
  \ifx\SB@next\SB@srspace\else%
    \nobreak\thinspace%
  \fi%
}
\newcommand\SB@srhyphen{\futurelet\SB@next\SB@@srhyphen}
\newcommand\SB@@srhyphen{%
  \ifx\SB@next\SB@srhyphen\expandafter\SB@srdash\else--\fi%
}
\newcommand\SB@srdash[1]{\futurelet\SB@next\SB@@srdash}
\newcommand\SB@@srdash{%
  \ifx\SB@next\SB@srhyphen---\expandafter\@gobble\else--\fi%
}
\newcommand\SB@srspace{\futurelet\SB@next\SB@@srspace}
\newcommand\SB@@srspace{%
  \let\SB@donext\relax%
  \ifx\SB@next\SB@srspace\else%
    \ifnum\spacefactor>\@m%
      \ifnum\spacefactor>1499 %
        \ifcat\noexpand\SB@next0%
          \let\SB@donext\SB@srcso%
        \else%
          \penalty-5\enskip%
        \fi%
      \else%
        \space%
      \fi%
    \else%
      \nobreak\space%
    \fi%
  \fi%
  \SB@donext%
}
\newcommand\SB@srcso[1]{\futurelet\SB@temp\SB@@srcso}
\newcommand\SB@@srcso{%
  \ifx\SB@temp\SB@srspace%
    \penalty-5\enskip%
  \else%
    \space%
  \fi%
  \SB@next%
}
\newif\ifSB@stanza
\newcommand\SB@stanzabreak{%
  \ifhmode\par\fi%
  \ifSB@stanza%
    \SB@breakpoint{%
      \ifSB@inverse%
        \ifSB@prevverse\vvpenalty\else\cvpenalty\fi%
      \else%
        \ifSB@prevverse\vcpenalty\else\ccpenalty\fi%
      \fi%
    }%
    \vskip\versesep%
  \fi%
}
\newcommand\SB@breakpoint[1]{%
  \begingroup%
    \ifnum#1<\@M%
      \SB@skip\colbotglue\relax%
      \SB@skip-\SB@skip%
    \else%
      \SB@skip\z@skip%
    \fi%
    \advance\SB@skip\lastskip%
    \unskip%
    \nobreak%
    \ifnum#1<\@M%
      \vskip\colbotglue\relax%
      \penalty#1%
    \fi%
    \vskip\SB@skip%
  \endgroup%
}
\newcommand\SB@putbox[2]{%
  \begingroup%
    \SB@dimen\dp#2%
    #1#2%
    \setbox\SB@box\hbox{{\lyricfont\relax p}}%
    \ifdim\SB@dimen<\dp\SB@box%
      \advance\SB@dimen-\dp\SB@box%
      \vskip-\SB@dimen%
    \fi%
    \setbox\SB@box\box\voidb@x%
  \endgroup%
}
\newcommand\SB@obeylines{%
  \let\par\SB@par%
  \obeylines%
  \let\@par\SB@@par%
}
\newcommand\SB@par{%
  \ifhmode%
    \SB@cnt\interlinepenalty%
    \interlinepenalty\@M%
    \endgraf%
    \interlinepenalty\SB@cnt%
    \ifSB@inchorus%
      \ifdim\cbarwidth>\z@\nobreak\else\SB@ilpenalty\fi%
    \else%
      \SB@ilpenalty%
    \fi%
  \fi%
}
\newcommand\SB@ilpenalty{%
  \ifnum\interlinepenalty=\z@\else%
    \penalty\interlinepenalty%
  \fi%
}
\newcommand\SB@@par{\let\par\SB@par\par}
\SB@newdimen\SB@parindent
\newcommand\SB@everypar{}
\newcommand\SB@raggedright{%
  \SB@parindent\parindent%
  \raggedright%
  \parindent\SB@parindent%
}
\newif\ifvnumbered
\newif\ifSB@prevverse
\CheckCommand\verse{%
  \let\\\@centercr%
  \list{}{%
    \itemsep\z@%
    \itemindent-1.5em%
    \listparindent\itemindent%
    \rightmargin\leftmargin%
    \advance\leftmargin1.5em%
  }%
  \item\relax%
}
\renewenvironment{verse}
  {\vnumberedtrue\SB@beginverse}
  {\SB@endverse}
\newenvironment{verse*}
  {\vnumberedfalse\SB@beginverse}
  {\SB@endverse}
\newcommand\beginverse{%
  \begingroup%
    \SB@loadactives%
    \@ifstar{\endgroup\vnumberedfalse\SB@beginverse}%
            {\endgroup\vnumberedtrue\SB@beginverse}%
}
\newcommand\SB@beginverse{%
  \ifSB@insong%
    \ifSB@inverse\SB@errbvv\endverse\fi%
    \ifSB@inchorus\SB@errbvc\endchorus\fi%
  \else%
    \SB@errbvt\beginsong{Unknown Song}%
  \fi%
  \ifrepchorus\ifvoid\SB@chorusbox\else%
    \SB@gotchorustrue%
    \ifSB@prevverse\ifvnumbered%
      \marks\SB@cmarkclass{\SB@cmark}%
    \fi\fi%
  \fi\fi%
  \SB@inversetrue%
  \def\SB@closeall{\endverse\endsong}%
  \SB@stanzabreak%
  \versemark\nobreak%
  \global\SB@stanzatrue%
  \SB@ifempty\SB@cr@\memorize{\replay[]}%
  \setbox\SB@box\vbox\bgroup\begingroup%
    \ifvnumbered%
      \protected@edef\@currentlabel{\p@versenum\theversenum}%
      \def\SB@everypar{%
        \setbox\SB@box\hbox{{\printversenum{\theversenum}}}%
        \ifdim\wd\SB@box<\versenumwidth%
          \setbox\SB@box%
          \hbox to\versenumwidth{\unhbox\SB@box\hfil}%
        \fi%
        \ifchorded\vrule\@height\baselineskip\@width\z@\@depth\z@\fi%
        \placeversenum\SB@box%
        \gdef\SB@everypar{}%
      }%
    \else%
      \def\SB@everypar{%
        \ifchorded\vrule\@height\baselineskip\@width\z@\@depth\z@\fi%
        \gdef\SB@everypar{}%
      }%
    \fi%
    \everypar{\SB@everypar\everypar{}}%
    \versefont\relax\SB@setbaselineskip\versejustify%
    \SB@loadactives%
    \SB@obeylines%
    \penalty12345 %
    \everyverse\relax%
}
\newcommand\SB@endverse{%
  \ifSB@insong%
    \ifSB@inverse%
        \unpenalty%
      \endgroup\egroup%
      \SB@putbox\unvbox\SB@box%
      \SB@inversefalse%
      \def\SB@closeall{\endsong}%
      \ifvnumbered\stepcounter{versenum}\fi%
      \SB@prevversetrue%
    \else\ifSB@inchorus\SB@errevc\endchorus%
    \else\SB@errevo\fi\fi%
  \else%
    \SB@errevt%
  \fi%
}
\newif\ifSB@chorustop
\SB@newbox\SB@chorusbox
\newif\ifSB@gotchorus
\ifSB@etex
  \@ifundefined{newmarks}{
    \@ifundefined{newmark}{
      \mathchardef\SB@cmarkclass83
      \mathchardef\SB@nocmarkclass84
    }{
      \newmark\SB@cmarkclass
      \newmark\SB@nocmarkclass
    }
  }{
    \newmarks\SB@cmarkclass
    \newmarks\SB@nocmarkclass
  }
\fi
\newcommand\SB@cmark{}
\def\SB@cmark{SB@cmark}
\newcommand\SB@lastcmark{}
\def\SB@lastcmark{SB@lastcmark}
\newcommand\SB@nocmark{}
\def\SB@nocmark{SB@nocmark}
\newenvironment{chorus}{\beginchorus}{\SB@endchorus}
\newcommand\beginchorus{%
  \ifSB@insong
    \ifSB@inverse\SB@errbcv\endverse\fi%
    \ifSB@inchorus\SB@errbcc\endchorus\fi%
  \else%
    \SB@errbct\beginsong{Unknown Song}%
  \fi%
  \SB@inchorustrue%
  \def\SB@closeall{\endchorus\endsong}%
  \SB@chorustoptrue%
  \vnumberedfalse%
  \SB@stanzabreak%
  \chorusmark%
  \ifrepchorus%
    \ifSB@gotchorus\else\ifSB@prevverse\else%
      \global\setbox\SB@chorusbox\vbox{%
        \unvbox\SB@chorusbox%
        \SB@stanzabreak%
        \chorusmark%
      }%
    \fi\fi%
  \fi%
  \global\SB@stanzatrue%
  \replay[]%
  \SB@@beginchorus%
  \everychorus\relax%
}
\newcommand\SB@@beginchorus{%
  \ifrepchorus\marks\SB@cmarkclass{}\fi%
  \setbox\SB@box\vbox\bgroup\begingroup%
    \ifchorded%
      \def\SB@everypar{%
        \vrule\@height\baselineskip\@width\z@\@depth\z@%
        \gdef\SB@everypar{}%
      }%
      \everypar{\SB@everypar\everypar{}}%
    \fi%
    \chorusfont\relax\SB@setbaselineskip\chorusjustify%
    \SB@loadactives%
    \SB@obeylines%
    \penalty12345 %
}
\newcommand\SB@endchorus{%
  \ifSB@insong%
    \ifSB@inchorus%
        \unpenalty%
      \endgroup\egroup%
      \SB@inchorusfalse%
      \def\SB@closeall{\endsong}%
      \setbox\SB@box\vbox{%
        \SB@chorusbar\SB@box%
        \SB@putbox\unvbox\SB@box%
      }
      \ifrepchorus\ifSB@gotchorus\else%
        \global\setbox\SB@chorusbox\vbox{%
          \unvbox\SB@chorusbox%
          \unvcopy\SB@box%
        }%
      \fi\fi%
      \unvbox\SB@box%
      \SB@prevversefalse%
    \else\ifSB@inverse\SB@errecv\endverse%
    \else\SB@erreco\fi\fi%
  \else%
    \SB@errect%
  \fi%
}
\newcommand\SB@cbarshift{%
  \ifSB@inchorus\ifdim\cbarwidth>\z@%
    \advance\leftskip\cbarwidth%
    \advance\leftskip5\p@\relax%
  \fi\fi%
}
\newcommand\SB@chorusbar[1]{%
  \ifdim\cbarwidth>\z@%
    \SB@dimen\ht#1%
    \SB@dimenii\dp#1%
    \advance\SB@dimen%
      \ifSB@chorustop\ifchorded\else2\fi\fi\SB@dimenii%
    \SB@skip\SB@dimen\relax%
    \SB@computess\SB@skip1\@plus#1%
    \SB@computess\SB@skip{-1}\@minus#1%
    \nointerlineskip\null\nobreak%
    \leaders\vrule\@width\cbarwidth\vskip\SB@skip%
    \ifSB@chorustop\ifchorded\else%
      \advance\SB@skip-\SB@dimenii%
    \fi\fi%
    \nobreak\vskip-\SB@skip%
  \fi%
}
\newcommand\SB@computess[4]{%
  \begingroup%
    \vbadness\@M\vfuzz\maxdimen%
    \SB@dimen4096\p@%
    \setbox\SB@box\vbox spread#2\SB@dimen{\unvcopy#4}%
    \ifnum\badness=\z@%
      \global\advance#1\z@#31fil\relax%
    \else%
      \SB@dimenii\SB@dimen%
      \loop%
        \SB@dimenii.5\SB@dimenii%
        \ifnum\badness<100 %
          \advance\SB@dimen\SB@dimenii%
        \else
          \advance\SB@dimen-\SB@dimenii%
        \fi%
        \setbox\SB@box\vbox spread#2\SB@dimen{\unvcopy#4}%
        \ifnum\badness=100 \SB@dimenii\z@\fi%
      \ifdim\SB@dimenii>.1\p@\repeat%
      \ifdim\SB@dimen<.1\p@\SB@dimen\z@\fi%
      \global\advance#1\z@#3\SB@dimen\relax%
    \fi%
  \endgroup%
}
\newcommand\brk{%
  \ifSB@insong%
    \ifhmode\penalty-5 \else%
      \unpenalty%
      \ifSB@inchorus%
        \ifdim\cbarwidth=\z@%
          \ifrepchorus\marks\SB@cmarkclass{}\fi%
          \SB@breakpoint\brkpenalty%
        \else%
          \endgroup\egroup%
          \ifrepchorus\ifSB@gotchorus\else%
            \global\setbox\SB@chorusbox\vbox{%
              \unvbox\SB@chorusbox%
              \SB@chorusbar\SB@box%
              \unvcopy\SB@box%
              \SB@breakpoint\brkpenalty%
            }%
          \fi\fi%
          \SB@chorusbar\SB@box%
          \unvbox\SB@box%
          \SB@breakpoint\brkpenalty%
          \SB@chorustopfalse%
          \SB@@beginchorus%
        \fi%
      \else%
        \SB@breakpoint\brkpenalty%
      \fi%
    \fi%
  \else%
    \ifpartiallist\else\SB@nextcol\@ne\colbotglue\fi%
  \fi%
}
\newcommand\SB@boxup[1]{%
  \setbox\SB@box\hbox{{\notefont\relax#1}}%
  \SB@dimen\wd\SB@box%
  \advance\SB@dimen6\p@%
  \advance\SB@dimen\leftskip%
  \advance\SB@dimen\rightskip%
  \ifdim\SB@dimen>\hsize%
    \vbox{{%
      \advance\hsize-6\p@%
      \advance\hsize-\leftskip%
      \advance\hsize-\rightskip%
      \notejustify%
      \unhbox\SB@box\par%
      \kern\z@%
    }}%
  \else%
    \vbox{\box\SB@box\kern\z@}%
  \fi%
}
\newcommand\textnote[1]{%
  \ifhmode\par\fi%
  \ifnum\lastpenalty=12345\else%
    \ifSB@inverse%
      \vskip2\p@\relax%
    \else\ifSB@inchorus%
      \vskip2\p@\relax%
    \else\ifSB@stanza%
      \nobreak\vskip\versesep%
    \fi\fi\fi%
  \fi%
  \begingroup%
    \everypar{}%
    \ifchorded\chordedfalse\SB@setbaselineskip\chordedtrue\fi%
    \placenote{\SB@colorbox\notebgcolor{\SB@boxup{#1}}}%
  \endgroup%
  \nobreak%
  \ifSB@inverse%
    \vskip2\p@\relax%
  \else\ifSB@inchorus%
    \vskip2\p@\relax%
  \else\ifSB@stanza\else%
    \nobreak\vskip\versesep%
  \fi\fi\fi%
}
\newcommand\musicnote[1]{\ifchorded\textnote{#1}\fi}
\ifSB@etex
  \newcommand\echo{\begingroup\@sanitize\SB@echo}
  \newcommand\SB@echo[1]{%
    \endgroup%
    \begingroup%
      \echofont\relax%
      \endlinechar\m@ne%
      \scantokens{(#1)}%
    \endgroup%
  }
\else
  \newcommand\echo{%
    \begingroup%
      \echofont\relax%
      \afterassignment\SB@echo%
      \setbox\SB@box\hbox%
  }
  \newcommand\SB@echo{\aftergroup\SB@@echo(}
  \newcommand\SB@@echo{\unhbox\SB@box)\endgroup}
\fi
\newcommand\rep[1]{%
  (\raise.25ex\hbox{%
    \fontencoding{OMS}\fontfamily{cmsy}\selectfont\char\tw@%
   }#1)%
}
\newenvironment{songgroup}{%
  \ifnum\SB@grouplvl=\z@%
    \edef\SB@sgroup{\thesongnum}%
    \global\SB@groupcnt\m@ne%
  \fi%
  \advance\SB@grouplvl\@ne%
}{%
  \advance\SB@grouplvl\m@ne%
  \ifnum\SB@grouplvl=\z@\let\SB@sgroup\@empty\fi%
}
\SB@newcount\SB@grouplvl
\newenvironment{intersong}{%
  \ifSB@insong\SB@errbro\SB@closeall\fi%
  \ifSB@intersong\SB@errbrr\SB@closeall\fi%
  \setbox\SB@chorusbox\box\voidb@x%
  \SB@intersongtrue%
  \def\SB@closeall{\end{intersong}}%
  \setbox\SB@songbox\vbox\bgroup\begingroup%
    \ifnum\SB@numcols>\z@\hsize\SB@colwidth\fi%
    \ifdim\sbarheight>\z@%
      \hrule\@height\sbarheight\@width\hsize%
      \nobreak%
    \fi%
}{%
  \ifSB@intersong
      \ifdim\sbarheight>\z@%
        \ifhmode\par\fi%
        \SB@skip\lastskip%
        \unskip\nobreak\vskip\SB@skip%
        \hbox{\vrule\@height\sbarheight\@width\hsize}%
      \fi%
    \endgroup\egroup%
    \ifSB@omitscrip%
      \setbox\SB@songbox\box\voidb@x%
    \else%
      \SB@submitsong%
    \fi%
    \SB@intersongfalse%
  \else%
    \ifSB@insong\SB@errero\SB@closeall\else\SB@errert\fi%
  \fi%
}
\newenvironment{intersong*}{%
  \ifSB@insong\SB@errbro\SB@closeall\fi%
  \ifSB@intersong\SB@errbrr\SB@closeall\fi%
  \setbox\SB@chorusbox\box\voidb@x%
  \SB@intersongtrue%
  \def\SB@closeall{\end{intersong*}}%
  \setbox\SB@songbox\vbox\bgroup\begingroup%
}{%
  \ifSB@intersong%
    \endgroup\egroup%
    \ifSB@omitscrip%
      \setbox\SB@songbox\box\voidb@x%
    \else%
      \def\SB@stype{\SB@styppage}%
      \SB@submitsong%
      \def\SB@stype{\SB@stypcol}%
    \fi%
    \SB@intersongfalse%
  \else%
    \ifSB@insong\SB@errero\SB@closeall\else\SB@errert\fi%
  \fi%
}
\newenvironment{scripture}{\beginscripture}{\SB@endscripture}
\newcommand\beginscripture[1]{%
  \begin{intersong}%
    \SB@parsesrefs{#1}%
    \setbox\SB@envbox\hbox{{\printscrcite\songrefs}}%
    \def\SB@closeall{\endscripture}%
    \nobreak\vskip5\p@%
    \SB@parindent\parindent\parindent\z@%
    \parskip\z@skip\parfillskip\@flushglue%
    \leftskip\SB@parindent\rightskip\SB@parindent\relax%
    \scripturefont\relax%
    \baselineskip\f@size\p@\@plus\p@\relax%
    \advance\baselineskip\p@\relax%
    \emergencystretch.3em%
}
\newcommand\SB@endscripture{%
  \ifSB@intersong
      \scitehere%
      \ifhmode\par\fi%
      \vskip-3\p@%
    \end{intersong}%
  \fi%
}
\newcommand\scitehere{%
  \ifSB@intersong%
    \ifvoid\SB@envbox\else%
      \ifvmode%
        \setbox\SB@box\lastbox%
        \nointerlineskip\noindent\hskip-\leftskip%
        \unhbox\SB@box\unskip%
      \fi%
      \unskip\nobreak\hfil\penalty50\hskip.8em\null\nobreak\hfil%
      \box\SB@envbox\kern-\SB@parindent%
      {\parfillskip\z@\finalhyphendemerits2000\par}%
    \fi%
  \else%
    \SB@errscrip\scitehere%
  \fi%
}
\newcommand\Acolon{\SB@colon2\Acolon}
\newcommand\Bcolon{\SB@colon1\Bcolon}
\newcommand\SB@colon[2]{%
  \ifSB@intersong\else%
    \SB@errscrip#2%
    \beginscripture{Unknown}%
  \fi%
  \ifhmode\par\fi%
  \begingroup%
    \rightskip\SB@parindent\@plus4em%
    \advance\leftskip2\SB@parindent%
    \advance\parindent-#1\SB@parindent%
    \def\par{\endgraf\endgroup}%
    \obeylines%
}
\newcommand\strophe{%
  \ifSB@intersong\else%
    \SB@errscrip\strophe\beginscripture{Unknown}%
  \fi%
  \vskip.9ex\@plus.45ex\@minus.68ex\relax%
}
\newcommand\SB@scripdent[2]{%
  \ifSB@intersong\else%
    \SB@errscrip#2\beginscripture{Unknown}%
  \fi%
  \ifhmode\par\fi%
  \advance\leftskip#1\SB@parindent\relax%
}
\newcommand\scripindent{\SB@scripdent1\scripindent}
\newcommand\scripoutdent{\SB@scripdent-\scripoutdent}
\newcommand\SB@quotesactive{%
  \catcode`'\active%
  \catcode``\active%
}
\newcommand\shiftdblquotes[4]{}
\newcommand\SB@ldqleft{}
\newcommand\SB@ldqright{}
\newcommand\SB@rdqleft{}
\newcommand\SB@rdqright{}
\newcommand\SB@scanlq{}
\newcommand\SB@scanrq{}
\newcommand\SB@dolq{}
\newcommand\SB@dorq{}
{
  \SB@quotesactive
  \gdef\shiftdblquotes#1#2#3#4{%
    \def\SB@ldqleft{\kern#1}%
    \def\SB@ldqright{\kern#2}%
    \def\SB@rdqleft{\kern#3}%
    \def\SB@rdqright{\kern#4}%
    \SB@quotesactive%
    \def`{\futurelet\SB@next\SB@scanlq}%
    \def'{\futurelet\SB@next\SB@scanrq}%
  }
  \gdef\SB@scanlq{%
    \ifx\SB@next`%
      \expandafter\SB@dolq%
    \else%
      \expandafter\lq%
    \fi%
  }
  \gdef\SB@scanrq{%
    \ifx\SB@next'%
      \expandafter\SB@dorq%
    \else%
      \expandafter\rq%
    \fi%
  }
  \gdef\SB@dolq`{%
    \ifvmode\leavevmode\else\/\fi%
    \vadjust{}%
    \SB@ldqleft\lq\lq\SB@ldqright%
    \vadjust{}%
  }
  \gdef\SB@dorq'{%
    \ifvmode\leavevmode\else\/\fi%
    \vadjust{}%
    \SB@rdqleft\rq\rq\SB@rdqright%
    \vadjust{}%
  }
}
\SB@newcount\SB@transposefactor
\newif\ifSB@convertnotes
\newcommand\notenameA{}
\newcommand\notenameB{}
\newcommand\notenameC{}
\newcommand\notenameD{}
\newcommand\notenameE{}
\newcommand\notenameF{}
\newcommand\notenameG{}
\newcommand\printnoteA{}
\newcommand\printnoteB{}
\newcommand\printnoteC{}
\newcommand\printnoteD{}
\newcommand\printnoteE{}
\newcommand\printnoteF{}
\newcommand\printnoteG{}
\newcommand\notenamesin[7]{%
  \def\notenameA{#1}%
  \def\notenameB{#2}%
  \def\notenameC{#3}%
  \def\notenameD{#4}%
  \def\notenameE{#5}%
  \def\notenameF{#6}%
  \def\notenameG{#7}%
  \SB@convertnotestrue%
}
\newcommand\notenamesout[7]{%
  \def\printnoteA{#1}%
  \def\printnoteB{#2}%
  \def\printnoteC{#3}%
  \def\printnoteD{#4}%
  \def\printnoteE{#5}%
  \def\printnoteF{#6}%
  \def\printnoteG{#7}%
  \SB@convertnotestrue%
}
\newcommand\notenames[7]{%
  \notenamesin{#1}{#2}{#3}{#4}{#5}{#6}{#7}%
  \notenamesout{#1}{#2}{#3}{#4}{#5}{#6}{#7}%
  \SB@convertnotesfalse%
}
\newcommand\alphascale{\notenames ABCDEFG}
\newcommand\solfedge{\notenames{LA}{SI}{DO}{RE}{MI}{FA}{SOL}}
\alphascale
\newif\ifSB@prefshrps
\newif\ifSB@needkey
\newcommand\transpose[1]{%
  \advance\SB@transposefactor by#1\relax%
  \SB@cnt\SB@transposefactor%
  \divide\SB@cnt12 %
  \multiply\SB@cnt12 %
  \advance\SB@transposefactor-\SB@cnt%
  \SB@needkeytrue%
}
\newcommand\capo[1]{%
  \iftranscapos\transpose{#1}\else\musicnote{capo #1}\fi%
}
\newcommand\prefersharps{\SB@prefshrpstrue\SB@needkeyfalse}
\newcommand\preferflats{\SB@prefshrpsfalse\SB@needkeyfalse}
\newcommand\transposehere[1]{%
  \ifnum\SB@transposefactor=\z@%
    \ifSB@convertnotes%
      \SB@dotranspose{#1}%
      \the\SB@toks%
    \else%
      #1%
    \fi%
  \else%
    \ifSB@convertnotes%
      {\SB@transposefactor\z@%
       \SB@dotranspose{#1}%
       \xdef\SB@tempv{\the\SB@toks}}%
    \else%
      \def\SB@tempv{#1}%
    \fi%
    \SB@dotranspose{#1}%
    \expandafter\trchordformat\expandafter{\SB@tempv}{\the\SB@toks}%
  \fi%
}
\newcommand\notrans[1]{%
  \begingroup%
    \SB@transposefactor\z@%
    \transposehere{#1}%
  \endgroup%
}
\newcommand\SB@dotranspose[1]{%
  \SB@toks{}%
  \let\SB@dothis\SB@trmain%
  \SB@trscan#1\SB@trend%
}
\newcommand\trchordformat[2]{#2}
\newcommand\SB@trscan{\futurelet\SB@next\SB@dothis}
\newcommand\SB@trmain{%
  \ifx\SB@next\bgroup%
    \let\SB@donext\SB@trgroup%
  \else\ifx\SB@next\egroup%
    \SB@toks\expandafter{\the\SB@toks\egroup}%
    \let\SB@donext\SB@trskip%
  \else\ifcat\noexpand\SB@next\@sptoken%
    \SB@appendsp\SB@toks%
    \let\SB@donext\SB@trskip%
  \else%
    \let\SB@donext\SB@trstep%
  \fi\fi\fi%
  \SB@donext%
}
\newcommand\SB@trgroup[1]{%
  \SB@toks\expandafter{\the\SB@toks{#1}}%
  \SB@trscan%
}
\newcommand\SB@trskip{%
  \afterassignment\SB@trscan%
  \let\SB@next= }
\newcommand\SB@trstep[1]{%
  \let\SB@donext\SB@trscan%
  \ifcat\noexpand\SB@next A%
    \ifnum\uccode`#1=`#1%
      \def\SB@temp{#1}%
      \let\SB@dothis\SB@trnote%
    \else%
      \SB@toks\expandafter{\the\SB@toks#1}%
    \fi%
  \else\ifx\SB@next\SB@trend
    \let\SB@donext\relax%
  \else%
    \SB@toks\expandafter{\the\SB@toks#1}%
  \fi\fi%
  \SB@donext%
}
\newcommand\SB@trnote{%
  \ifcat\noexpand\SB@next A%
    \let\SB@donext\SB@trnotestep%
  \else\ifnum\SB@transposefactor=\z@%
    \SB@cnt\z@%
    \let\SB@donext\SB@trtrans%
  \else\ifx\SB@next\flt%
    \SB@cnt\m@ne%
    \let\SB@donext\SB@tracc%
  \else\ifx\SB@next\shrp%
    \SB@cnt\@ne%
    \let\SB@donext\SB@tracc%
  \else%
    \SB@cnt\z@%
    \let\SB@donext\SB@trtrans%
  \fi\fi\fi\fi%
  \SB@donext%
}
\newcommand\SB@trnotestep[1]{%
  \ifnum\uccode`#1=`#1%
    \SB@app\def\SB@temp{#1}%
    \expandafter\SB@trscan%
  \else%
    \SB@cnt\z@%
    \expandafter\SB@trtrans\expandafter#1%
  \fi%
}
\newcommand\SB@tracc[1]{\futurelet\SB@next\SB@trtrans}
\newcommand\SB@trtrans{%
  \advance\SB@cnt%
    \ifx\SB@temp\notenameA\z@%
    \else\ifx\SB@temp\notenameB\tw@%
    \else\ifx\SB@temp\notenameC\thr@@%
    \else\ifx\SB@temp\notenameD5 %
    \else\ifx\SB@temp\notenameE7 %
    \else\ifx\SB@temp\notenameF8 %
    \else\ifx\SB@temp\notenameG10 %
    \else-99 \fi\fi\fi\fi\fi\fi\fi%
  \ifnum\SB@cnt<\m@ne%
    \SB@toks\expandafter\expandafter\expandafter{%
      \expandafter\the\expandafter\SB@toks\SB@temp}%
  \else%
    \advance\SB@cnt\SB@transposefactor%
    \ifnum\SB@cnt<\z@\advance\SB@cnt12 \fi%
    \ifnum\SB@cnt>11 \advance\SB@cnt-12 \fi%
    \ifSB@needkey\ifnum\SB@transposefactor=\z@\else\SB@setkeysig\fi\fi%
    \edef\SB@temp{%
      \the\SB@toks%
      \ifSB@prefshrps%
        \ifcase\SB@cnt\printnoteA\or\printnoteA\noexpand\shrp\or%
          \printnoteB\or\printnoteC\or\printnoteC\noexpand\shrp\or%
          \printnoteD\or\printnoteD\noexpand\shrp\or\printnoteE\or%
          \printnoteF\or\printnoteF\noexpand\shrp\or\printnoteG\or%
          \printnoteG\noexpand\shrp\fi%
      \else%
        \ifcase\SB@cnt\printnoteA\or\printnoteB\noexpand\flt\or%
          \printnoteB\or\printnoteC\or\printnoteD\noexpand\flt\or%
          \printnoteD\or\printnoteE\noexpand\flt\or\printnoteE\or%
          \printnoteF\or\printnoteG\noexpand\flt\or\printnoteG\or%
          \printnoteA\noexpand\flt\fi%
      \fi}%
    \SB@toks\expandafter{\SB@temp}%
  \fi%
  \let\SB@dothis\SB@trmain%
  \SB@trscan%
}
\newcommand\SB@setkeysig{%
  \global\SB@needkeyfalse%
  \ifcase\SB@cnt%
    \global\SB@prefshrpstrue\or% A
    \global\SB@prefshrpsfalse\or% Bb
    \global\SB@prefshrpstrue\or% B
    \ifx\SB@next m% C
      \global\SB@prefshrpsfalse%
    \else%
      \global\SB@prefshrpstrue%
    \fi\or%
    \global\SB@prefshrpstrue\or% C#
    \ifx\SB@next m% D
      \global\SB@prefshrpsfalse%
    \else%
      \global\SB@prefshrpstrue%
    \fi\or%
    \global\SB@prefshrpsfalse\or% Eb
    \global\SB@prefshrpstrue\or% E
    \global\SB@prefshrpsfalse\or% F
    \global\SB@prefshrpstrue\or% F#
    \ifx\SB@next m% G
      \global\SB@prefshrpsfalse%
    \else%
      \global\SB@prefshrpstrue%
    \fi\or%
    \global\SB@prefshrpsfalse\else% Ab
    \global\SB@needkeytrue% non-chord
  \fi%
}
\newcommand\SB@trend{%
  \SB@Error{Internal Error: Transposition failed}%
           {This error should not occur.}%
}
\newcommand\SB@metertop{}
\newcommand\SB@meterbot{}
\newcommand\meter[2]{\gdef\SB@metertop{#1}\gdef\SB@meterbot{#2}}
\newcommand\SB@measuremark{SB@IsMeasure}
\newcommand\SB@makembar[2]{%
  \ifSB@inverse\else%
    \ifSB@inchorus\else\SB@errmbar\fi%
  \fi%
  \ifhmode%
    \SB@skip\lastskip\unskip%
    \setbox\SB@box\lastbox%
    \copy\SB@box%
    \ifvbox\SB@box%
      \begingroup%
        \setbox\SB@boxii\copy\SB@box%
        \vbadness\@M\vfuzz\maxdimen%
        \setbox\SB@boxii%
          \vsplit\SB@boxii to\maxdimen%
      \endgroup%
      \long\edef\SB@temp{\splitfirstmark}%
      \ifx\SB@temp\SB@measuremark%
        \penalty100\hskip1em%
      \else%
        \penalty100\hskip\SB@skip%
      \fi%
    \else%
      \penalty100\hskip\SB@skip%
    \fi%
  \fi%
  \ifvmode\leavevmode\fi%
  \setbox\SB@box\hbox{{\meterfont\relax#1}}%
  \setbox\SB@boxii\hbox{{\meterfont\relax#2}}%
  \SB@dimen\wd\ifdim\wd\SB@box>\wd\SB@boxii\SB@box\else\SB@boxii\fi%
  \SB@dimenii\baselineskip%
  \advance\SB@dimenii-2\p@%
  \advance\SB@dimenii-\ht\SB@box%
  \advance\SB@dimenii-\dp\SB@box%
  \advance\SB@dimenii-\ht\SB@boxii%
  \advance\SB@dimenii-\dp\SB@boxii%
  \let\SB@temp\relax%
  \ifdim\SB@dimen>\z@%
    \advance\SB@dimenii-.75\p@%
    \def\SB@temp{\kern.75\p@}%
  \fi%
  \SB@maxmin\SB@dimen<{.5\p@}%
  \SB@maxmin\SB@dimenii<\z@%
  \vbox{%
    \mark{\SB@measuremark}%
    \hbox to\SB@dimen{%
      \hfil%
      \box\SB@box%
      \hfil%
    }%
    \nointerlineskip%
    \hbox to\SB@dimen{%
      \hfil%
      \box\SB@boxii%
      \hfil%
    }%
    \SB@temp%
    \nointerlineskip%
    \hbox to\SB@dimen{%
      \hfil%
      \vrule\@width.5\p@\@height\SB@dimenii%
      \hfil%
    }%
  }%
  \meter{}{}%
}
\newcommand\mbar{\SB@mbar}
\newcommand\measurebar{%
  \mbar\SB@metertop\SB@meterbot%
}
\newcommand\SB@repcolon{{%
  \usefont{OT1}{cmss}{m}{n}\selectfont%
  \ifchorded%
    \baselineskip.5\SB@dimen%
    \vbox{\hbox{:}\hbox{:}\kern.5\p@}%
  \else%
    \raise.5\p@\hbox{:}%
  \fi%
}}
\newcommand\lrep{%
  \SB@dimen\baselineskip%
  \advance\SB@dimen-2\p@%
  \vrule\@width1.5\p@\@height\SB@dimen\@depth\p@%
  \kern1.5\p@%
  \vrule\@width.5\p@\@height\SB@dimen\@depth\p@%
  \SB@repcolon%
}
\newcommand\rrep{%
  \SB@dimen\baselineskip%
  \advance\SB@dimen-2\p@%
  \SB@repcolon%
  \vrule\@width.5\p@\@height\SB@dimen\@depth\p@%
  \kern1.5\p@%
  \vrule\@width1.5\p@\@height\SB@dimen\@depth\p@%
}
\newif\ifSB@wordends
\newif\ifSB@brokenword
\SB@newtoks\SB@lyric
\SB@newcount\SB@numhyps
\SB@newtoks\SB@lyricnohyp
\SB@newbox\SB@lyricbox
\SB@newbox\SB@chordbox
\newcommand\SB@chbstok{}
\newcommand\SB@setchord{}
{
  \catcode`^\active
  \gdef\SB@setchord#1{%
    \SB@gettabindtrue\SB@nohattrue%
    \setbox\SB@chordbox\hbox{%
      \unhbox\SB@chordbox%
      \begingroup%
        \ifSB@trackch%
          \let\SB@activehat\SB@hat@tr%
        \else%
          \let\SB@activehat\SB@hat@notr%
        \fi%
        \let^\SB@activehat%
        \printchord{%
          \ifSB@firstchord\else\kern.15em\fi%
          \vphantom/%
          \transposehere{#1}%
          \kern.2em%
        }%
      \endgroup%
    }%
    \SB@gettabindfalse%
    \ifSB@trackch\ifSB@nohat%
      \global\SB@creg\expandafter{\the\SB@creg#1\\}%
    \fi\fi%
    \let\SB@noreplay\@firstofone%
  }
}
\newcommand\SB@outertest{%
  \expandafter\SB@otesta\meaning\SB@next:\SB@otesta%
}
\newcommand\SB@otesta{}
\edef\SB@otesta#1:#2\SB@otesta{%
  \noexpand\SB@otestb%
  #1\string\outer%
  \noexpand\SB@otestb%
}
\newcommand\SB@otestb{}
\expandafter\def\expandafter\SB@otestb%
\expandafter#\expandafter1\string\outer#2\SB@otestb{%
  \def\SB@temp{#2}%
  \ifx\SB@temp\@empty\SB@testfalse\else\SB@testtrue\fi%
}
\newcommand\SB@UTFtest{}
\edef\SB@UTFtest#1{%
  \noexpand\expandafter%
  \noexpand\SB@@UTFtest%
  \noexpand\meaning#1%
  \string\UTFviii@zero@octets%
  \noexpand\SB@@UTFtest%
}
\newcommand\SB@U@two{\global\SB@cnt\tw@}
\newcommand\SB@U@three{\global\SB@cnt\thr@@}
\newcommand\SB@U@four{\global\SB@cnt4\relax}
\newcommand\SB@@UTFtest{}
{\escapechar\m@ne
 \xdef\SB@temp{\string\@octets}}
\edef\SB@temp{##1\string\UTFviii@##2\SB@temp}
\expandafter\def\expandafter\SB@@UTFtest\SB@temp#3\SB@@UTFtest{%
  \SB@cnt\z@%
  {\csname SB@U@#2\endcsname}%
}
\newcommand\SB@declare[3]{%
  \afterassignment\iffalse\let\SB@next= #3\relax\fi%
  \SB@UTFtest\SB@next%
  \ifcase\SB@cnt%
    \ifcat\noexpand#3\relax%
      \SB@addNtest\SB@macrotests#1#2#3%
    \else\ifcat\noexpand#3.%
      \SB@addDtest\SB@othertests#1#2%
    \else\ifcat\noexpand#3A%
      \SB@addDtest\SB@lettertests#1#2%
    \else%
      \SB@addDtest\relax0#2%
    \fi\fi\fi%
  \or%
    \SB@addNtest\SB@macrotests#1#2#3%
  \else%
    \SB@addMtest\SB@multitests#1#2#3\relax\relax\relax%
  \fi%
}
\newcommand\DeclareLyricChar{\SB@declare\SB@testtrue0}
\newcommand\DeclareNonLyric{\SB@declare\SB@testfalse\SB@testfalse}
\newcommand\DeclareNoHyphen{\SB@declare\SB@testfalse\SB@testtrue}
\newcommand\SB@lettertests{}
\newcommand\SB@macrotests{}
\newcommand\SB@multitests{}
\newcommand\SB@othertests{}
\newcommand\SB@addDtest[3]{%
  \ifx0#2\else%
    \def#1{{\csname SB@!\meaning\SB@next\endcsname}}%
    \expandafter\def\csname SB@!\meaning\SB@next\endcsname{\global#2}%
  \fi%
  \ifx0#3\else%
    \expandafter\def\csname SB@HT@\meaning\SB@next\endcsname{\global#3}%
  \fi%
}
\newcommand\SB@addNtest[4]{%
  \ifx0#2\else%
    \def#1{{\csname SB@!\SB@nextname\endcsname}}%
    \expandafter\def\csname SB@!\string#4\endcsname{\global#2}%
  \fi%
  \ifx0#3\else%
    \expandafter\def\csname SB@HT@\string#4\endcsname{\global#3}%
  \fi%
}
\newcommand\SB@addMtest[7]{%
  \edef\SB@temp{%
    \string#4%
    \ifx\relax#5\else\string#5\fi%
    \ifx\relax#6\else\string#6\fi%
    \ifx\relax#7\else\string#7\fi%
  }%
  \ifx0#2\else%
    \def#1{{\csname SB@!\SB@nextname\endcsname}}%
    \expandafter\def\csname SB@!\SB@temp\endcsname{\global#2}%
  \fi%
  \ifx0#3\else%
    \expandafter\def\csname SB@HT@\SB@temp\endcsname{\global#3}%
  \fi%
}
\DeclareLyricChar\`
\DeclareLyricChar\'
\DeclareLyricChar\^
\DeclareLyricChar\"
\DeclareLyricChar\~
\DeclareLyricChar\=
\DeclareLyricChar\.
\DeclareLyricChar\u
\DeclareLyricChar\v
\DeclareLyricChar\H
\DeclareLyricChar\t
\DeclareLyricChar\c
\DeclareLyricChar\d
\DeclareLyricChar\b
\DeclareLyricChar\oe
\DeclareLyricChar\OE
\DeclareLyricChar\ae
\DeclareLyricChar\AE
\DeclareLyricChar\aa
\DeclareLyricChar\AA
\DeclareLyricChar\o
\DeclareLyricChar\O
\DeclareLyricChar\l
\DeclareLyricChar\L
\DeclareLyricChar\ss
\DeclareLyricChar\i
\DeclareLyricChar\j
\DeclareLyricChar\/
\DeclareLyricChar\-
\DeclareLyricChar\discretionary
\DeclareNoHyphen\par
\newcommand\SB@bracket{}
\newcommand\SB@chord{\SB@begincname\SB@@chord}
\newcommand\SB@begincname{}
{\catcode`\^^M\active
 \gdef\SB@begincname{%
   \begingroup%
     \catcode`##\active\catcode`&\active%
     \catcode`:12\relax%
     \catcode`\^^M\active\SB@outer\def^^M{}%
     \SB@outer\def\[{}%
     \chordlocals\relax%
  }
}
\newcommand\SB@endcname{}
\let\SB@endcname\endgroup
\newcommand\SB@nbsp{}
\def\SB@nbsp{\nobreakspace{}}
\newif\ifSB@firstchord\SB@firstchordtrue
\newcommand*\SB@@chord{}
\def\SB@@chord#1]{%
  \SB@endcname%
  \ifSB@firstchord%
    \setbox\SB@lyricbox\hbox{\kern\SB@tabindent}%
    \global\SB@tabindent\z@%
    \SB@lyric{}%
    \SB@numhyps\z@%
    \SB@spcinit%
    \setbox\SB@chordbox\box\voidb@x%
  \fi%
  \SB@setchord{#1}%
  \SB@firstchordfalse%
  \let\SB@dothis\SB@chstart%
  \SB@chscan%
}
\newcommand\SB@spcinit{}
\newcommand\MultiwordChords{%
  \def\SB@spcinit{%
    \let\SB@chdone\SB@chlyrdone%
    \let\SB@chimpspace\SB@chnxtdone%
    \let\SB@chexpspace\SB@chnxtdone%
    \let\SB@chespace\SB@chendspace%
  }%
}
\newcommand\SB@chscan{%
  \let\SB@nextname\relax%
  \futurelet\SB@next\SB@chmain%
}
\newcommand\SB@chmain{\SB@dothis\SB@donext}
\newcommand\SB@chnxtrelax{\let\SB@donext\relax}
\newcommand\SB@chnxtstep{\let\SB@donext\SB@chstep}
\newcommand\SB@chnxtdone{\let\SB@donext\SB@chdone}
\newcommand\SB@chstart{%
  \ifx\SB@next\[\SB@chnxtrelax%
  \else\ifx\SB@next\SB@activehat\SB@chnxtrelax%
  \else\ifx\SB@next\ch\SB@chnxtrelax%
  \else\ifx\SB@next\mch\SB@chnxtrelax%
  \else\ifx\SB@next`\SB@chnxtstep%
  \else\ifx\SB@next'\SB@chnxtstep%
  \else\ifx\SB@next"\SB@chnxtstep%
  \else%
    \the\SB@lyric%
    \SB@lyric{}%
    \SB@firstchordtrue%
    \let\SB@dothis\SB@chnorm%
    \SB@chnorm%
  \fi\fi\fi\fi\fi\fi\fi%
}
\newcommand\SB@chnorm{%
  \ifcat\noexpand\SB@next A%
    \SB@testtrue\SB@lettertests%
    \ifSB@test%
      \SB@chespace\SB@chnxtstep%
    \else%
      \SB@chnxtdone%
    \fi%
  \else%
    \SB@chtrymacro%
  \fi%
}
\newcommand\SB@chtrymacro{%
  \ifcat\noexpand\SB@next\relax%
    \SB@chmacro%
  \else%
    \SB@chother%
  \fi%
}
\newcommand\SB@chother{%
  \ifcat\noexpand\SB@next\@sptoken%
    \SB@chexpspace%
  \else\ifcat\noexpand\SB@next\bgroup%
    \SB@chespace\let\SB@donext\SB@chbgroup%
  \else\ifcat\noexpand\SB@next\egroup%
    \SB@chespace\let\SB@donext\SB@chegroup%
  \else\ifx\SB@next-%
    \SB@numhyps\@ne\relax%
    \SB@lyricnohyp\expandafter{\the\SB@lyric}%
    \let\SB@dothis\SB@chhyph%
    \SB@chespace\SB@chnxtstep%
  \else\ifcat\noexpand\SB@next.%
    \SB@testtrue\SB@othertests%
    \ifSB@test%
      \SB@chespace\SB@chnxtstep%
    \else%
      \SB@chnxtdone%
    \fi%
  \else%
    \SB@chespace\SB@chnxtstep%
  \fi\fi\fi\fi\fi%
}
\newcommand\SB@chmacro{%
  \SB@outertest%
  \ifSB@test%
    \SB@chnxtdone%
  \else%
    \let\SB@donext\SB@chgetname%
  \fi%
}
\newcommand\SB@chgetname[1]{%
  \edef\SB@nextname{\string#1}%
  \SB@@chmacro\SB@donext#1%
}
\newcommand\SB@@chmacro{%
  \ifx\SB@next\SB@activehat%
    \SB@chnxtdone%
  \else\ifx\SB@next\SB@par%
    \SB@chnxtdone%
  \else\ifx\SB@next\measurebar%
    \SB@chmbar%
  \else\ifx\SB@next\mbar%
    \SB@chmbar%
  \else\ifx\SB@next\ch%
    \SB@chespace\let\SB@donext\SB@chlig%
  \else\ifx\SB@next\mch%
    \SB@chespace\let\SB@donext\SB@mchlig%
  \else\ifx\SB@next\ %
    \SB@chimpspace%
  \else\ifx\SB@next\SB@nbsp%
    \SB@chimpspace%
  \else%
    \SB@UTFtest\SB@next%
    \ifcase\SB@cnt\SB@chothermac%
    \or\or\SB@chespace\let\SB@donext\SB@chsteptwo%
    \or\SB@chespace\let\SB@donext\SB@chstepthree%
    \or\SB@chespace\let\SB@donext\SB@chstepfour\fi%
  \fi\fi\fi\fi\fi\fi\fi\fi%
}
\newcommand\SB@chothermac{%
  \SB@testfalse%
  \afterassignment\iffalse%
  \SB@cnt\expandafter`\SB@nextname x\fi%
  \ifnum\the\catcode\SB@cnt=\z@\else\SB@testtrue\fi%
  \SB@macrotests%
  \ifSB@test%
    \SB@chespace\SB@chnxtstep%
  \else%
    \SB@chnxtdone%
  \fi%
}
\newcommand\SB@chstep[1]{%
  \SB@lyric\expandafter{\the\SB@lyric#1}%
  \SB@chscan%
}
\newcommand\SB@chsteptwo[2]{\SB@chmulti{#1#2}{\string#1\string#2}}
\newcommand\SB@chstepthree[3]{%
  \SB@chmulti{#1#2#3}{\string#1\string#2\string#3}%
}
\newcommand\SB@chstepfour[4]{%
  \SB@chmulti{#1#2#3#4}{\string#1\string#2\string#3\string#4}%
}
\newcommand\SB@chmulti[2]{%
  \def\SB@next{#1}%
  \edef\SB@nextname{#2}%
  \SB@testtrue\SB@multitests%
  \ifSB@test%
    \SB@lyric\expandafter{\the\SB@lyric#1}%
    \expandafter\SB@chscan%
  \else%
    \expandafter\SB@chmstop%
  \fi%
}
\newcommand\SB@chmstop{\expandafter\SB@chdone\SB@next}
\newcommand\SB@chhyph{%
  \ifx\SB@next-%
    \advance\SB@numhyps\@ne\relax%
    \SB@chnxtstep%
  \else%
    \SB@chnxtdone%
  \fi%
}
\newcommand\SB@chimpspace{}
\let\SB@chimpspace\SB@chnxtdone
\newcommand\SB@chexpspace{}
\let\SB@chexpspace\SB@chnxtdone
\newcommand\SB@chespace{}
\newcommand\SB@chendspace{%
  \let\SB@chdone\SB@chlyrdone%
  \def\SB@chexpspace{\SB@chbspace\SB@chexpspace}%
  \def\SB@chimpspace{\SB@chbspace\SB@chimpspace}%
  \def\SB@chespace{}%
}
\newcommand\SB@chbspace{%
  \setbox\SB@lyricbox\hbox{%
    \unhbox\SB@lyricbox%
    \the\SB@lyric%
  }%
  \SB@lyric{}%
  \ifdim\wd\SB@lyricbox<\wd\SB@chordbox%
    \let\SB@chbstok= \SB@next%
    \def\SB@chexpspace{\let\SB@donext\SB@chgetspace}%
    \let\SB@chimpspace\SB@chnxtstep%
    \let\SB@chespace\SB@chendspace%
    \let\SB@chdone\SB@chspcdone%
  \else%
    \let\SB@chimpspace\SB@chnxtdone%
    \let\SB@chexpspace\SB@chnxtdone%
  \fi%
}
\newcommand\SB@chgetspace{%
  \SB@appendsp\SB@lyric%
  \let\SB@nextname\relax%
  \afterassignment\SB@chscan%
  \let\SB@next= }
\newcommand\SB@chmbar{%
  \ifmeasures%
    \SB@chnxtdone%
  \else%
    \SB@chespace\SB@chnxtstep%
  \fi%
}
\newcommand\SB@chbgroup[1]{%
  \SB@lyric\expandafter{\the\SB@lyric{#1}}%
  \SB@chscan%
}
\newcommand\SB@chegroup{%
  \let\SB@nextname\relax%
  \afterassignment\SB@chegrpscan%
  \let\SB@next= }
\newcommand\SB@chegrpscan{%
  \futurelet\SB@next\SB@chegrpmacro%
}
\newcommand\SB@chegrpmacro{%
  \ifcat\noexpand\SB@next\relax%
    \expandafter\SB@chegrpouter%
  \else%
    \expandafter\SB@chegrpdone%
  \fi%
}
\newcommand\SB@chegrpouter{%
  \SB@outertest%
  \ifSB@test%
    \expandafter\SB@chegrpdone%
  \else%
    \expandafter\SB@chegrpname%
  \fi%
}
\newcommand\SB@chegrpname[1]{%
  \edef\SB@nextname{\string#1}%
  \SB@chegrpdone#1%
}
\newcommand\SB@chegrpdone{\SB@chdone\egroup}
\newcommand\SB@chlig[5]{%
  \gdef\SB@ligpre{{#3}}%
  \gdef\SB@ligpost{\[#2]{#4}}%
  \gdef\SB@ligfull{%
    \[\SB@noreplay{\hphantom{{\lyricfont\relax#3}}}#2]{#5}%
  }%
  \SB@chdone%
}
\newcommand\SB@mchlig[5]{%
  \SB@lyric\expandafter{\the\SB@lyric#3}%
  \let\SB@next\measurebar%
  \edef\SB@nextname{\string\measurebar}%
  \gdef\SB@ligpost{\measurebar\[#2]{#4}}%
  \gdef\SB@ligfull{\measurebar\[#2]{#4}}%
  \SB@chdone%
}
\newcommand\SB@chlyrdone{%
  \setbox\SB@lyricbox\hbox{%
    \unhbox\SB@lyricbox%
    \ifnum\SB@numhyps=\@ne%
      \the\SB@lyricnohyp%
    \else%
      \the\SB@lyric%
    \fi%
  }%
  \SB@emitchord%
}
\newcommand\SB@chspcdone{%
  \let\SB@nextname\relax%
  \let\SB@next= \SB@chbstok%
  \expandafter\SB@emitchord\the\SB@lyric%
}
\newcommand\SB@chdone{}
\let\SB@chdone\SB@chlyrdone
\newcommand\SB@ligpre{}
\newcommand\SB@ligpost{}
\newcommand\SB@ligfull{}
\newcommand\SB@clearlig{%
  \gdef\SB@ligpre{}%
  \gdef\SB@ligpost{}%
  \gdef\SB@ligfull{}%
}
\newcommand\SB@emitchord{%
  \ifSB@inverse\else\ifSB@inchorus\else\SB@errchord\fi\fi%
  \SB@testfalse%
  \ifcat\noexpand\SB@next\@sptoken\SB@testtrue\fi%
  \ifcat\noexpand\SB@next.\SB@testtrue\fi%
  \ifx\SB@next\SB@par\SB@testtrue\fi%
  \ifx\SB@next\egroup\SB@testtrue\fi%
  \ifx\SB@next\endgroup\SB@testtrue\fi%
  {\csname%
     SB@HT@\ifx\SB@nextname\relax\meaning\SB@next\else\SB@nextname\fi%
   \endcsname}%
  \ifSB@test\SB@wordendstrue\else\SB@wordendsfalse\fi%
  \SB@dimen\wd\SB@chordbox%
  \ifvmode\leavevmode\fi%
  \SB@brokenwordfalse%
  \ifdim\wd\SB@lyricbox>\z@%
    \ifdim\SB@dimen>\wd\SB@lyricbox%
      \ifSB@wordends\else\SB@brokenwordtrue\fi%
    \fi%
  \else%
    \SB@skip\lastskip%
    \unskip\penalty200\hskip\SB@skip%
  \fi%
  \ifnum\SB@numhyps>\z@%
    \ifnum\SB@numhyps>\@ne%
      \SB@brokenwordfalse%
    \else%
      \SB@brokenwordtrue%
    \fi%
  \fi%
  \SB@testfalse%
  \ifnolyrics\ifdim\wd\SB@lyricbox=\z@\SB@testtrue\fi\fi%
  \ifSB@test%
    \unhbox\SB@chordbox%
    \gdef\SB@temp{\expandafter\SB@clearlig\SB@ligfull}%
  \else%
    \vbox{\clineparams\relax%
      \ifSB@brokenword%
        \global\setbox\SB@lyricbox\hbox{%
          \unhbox\SB@lyricbox%
          \SB@ligpre%
        }%
        \SB@maxmin\SB@dimen<{\wd\SB@lyricbox}%
        \advance\SB@dimen.5em%
        \hbox to\SB@dimen{\unhbox\SB@chordbox\hfil}%
        \hbox to\SB@dimen{%
          \unhcopy\SB@lyricbox\hfil
          \ifnum\hyphenchar\font>\m@ne\char\hyphenchar\font\hfil\fi%
        }%
        \global\SB@cnt\@m%
        \gdef\SB@temp{\expandafter\SB@clearlig\SB@ligpost}%
      \else%
        \box\SB@chordbox%
        \hbox{%
          \unhcopy\SB@lyricbox%
          \global\SB@cnt\spacefactor%
          \hfil%
        }%
        \gdef\SB@temp{\expandafter\SB@clearlig\SB@ligfull}%
      \fi%
    }%
    \ifSB@wordends%
      \ifdim\wd\SB@lyricbox>\z@\else\nobreak\fi%
    \else%
      \penalty%
        \ifnum\SB@numhyps>\z@\exhyphenpenalty%
        \else\ifSB@brokenword\hyphenpenalty%
        \else\@M\fi\fi%
      \spacefactor\SB@cnt%
    \fi%
  \fi%
  \SB@temp%
}
\newcommand\SB@accidental[1]{{%
  \/%
  \m@th#1%
  \SB@dimen-\fontdimen\@ne\font%
  \advance\SB@dimen.088142\p@%
  \ifdim\SB@dimen<\z@%
    \kern\f@size\SB@dimen%
  \fi%
}}
\newcommand\sharpsymbol{\ensuremath{^\#}}
\newcommand\flatsymbol{\raise.5ex\hbox{{\SB@flatsize$\flat$}}}
\newcommand\shrp{\SB@accidental\sharpsymbol}
\newcommand\flt{\SB@accidental\flatsymbol}
\newcommand\DeclareFlatSize[2]{%
  \expandafter\xdef\csname SB@flatsize@#1\endcsname{#2}%
}
\DeclareFlatSize\@vpt\@vpt
\DeclareFlatSize\@vipt\@vipt
\DeclareFlatSize\@viipt\@vipt
\DeclareFlatSize\@viiipt\@viipt
\DeclareFlatSize\@ixpt\@viiipt
\DeclareFlatSize\@xpt\@ixpt
\DeclareFlatSize\@xipt\@xpt
\DeclareFlatSize\@xiipt\@xipt
\DeclareFlatSize\@xivpt\@xiipt
\DeclareFlatSize\@xviipt\@xivpt
\DeclareFlatSize\@xxpt\@xviipt
\DeclareFlatSize\@xxvpt\@xxpt
\newcommand\SB@flatsize{%
  \@ifundefined{SB@flatsize@\f@size}{}{%
    \expandafter\fontsize%
      \csname SB@flatsize@\f@size\endcsname\f@baselineskip%
    \selectfont%
  }%
}
\newcommand\ch{\SB@ch}
\newcommand\SB@ch{}
\newcommand\SB@ch@on{\SB@begincname\SB@@ch}
\newcommand*\SB@@ch[1]{\SB@endcname\SB@@@ch{#1}}
\newcommand*\SB@@@ch[4]{\[\SB@noreplay{\hphantom{#2}}#1]#4}
\newcommand*\SB@ch@off[4]{#4}
\newcommand\mch{\SB@mch}
\newcommand\SB@mch{}
\newcommand*\SB@mch@m[4]{#2\measurebar#3}
\newcommand\SB@mch@on{\SB@begincname\SB@@mch}
\newcommand*\SB@@mch[1]{\SB@endcname\SB@@@mch{#1}}
\newcommand*\SB@@@mch[4]{#2\measurebar\[#1]#3}
\newcommand\SB@activehat{%
  \ifmmode^\else\expandafter\SB@rechord\fi%
}
\newcommand\SB@hat@tr{%
  \ifmmode^\else\global\SB@nohatfalse\fi%
}
\newcommand\SB@hat@notr{%
  \ifmmode^\else%
    \SB@lop\SB@ctail\SB@toks%
    \expandafter\transposehere\expandafter{\the\SB@toks}%
  \fi%
}
\newcommand\SB@loadactives{}
{
  \catcode`&\active
  \catcode`#\active
  \catcode`|\active
  \catcode`^\active
  \global\let&\flt
  \global\let#\shrp
  \global\let|\measurebar
  \global\let^\SB@activehat
  \gdef\SB@loadactives{%
    \catcode`^\ifchorded\active\else9 \fi%
    \catcode`|\ifmeasures\active\else9 \fi%
    \def\[{\SB@bracket}%
  }
}
\newif\ifSB@trackch
\SB@newtoks\SB@cr@
\SB@newtoks\SB@ctail
\newcommand\SB@creg{}
\newcommand\newchords[1]{%
  \@ifundefined{SB@cr@#1}{%
    \expandafter\SB@newtoks\csname SB@cr@#1\endcsname%
    \global\csname SB@cr@#1\endcsname{\\}%
  }{\SB@errdup{#1}}%
}
\newcommand\memorize{%
  \@ifnextchar[\SB@memorize{\SB@memorize[]}%
}
\newcommand\SB@memorize{}
\def\SB@memorize[#1]{%
  \@ifundefined{SB@cr@#1}{\SB@errreg{#1}}{%
    \SB@trackchtrue%
    \global\expandafter\let\expandafter\SB@creg%
      \csname SB@cr@#1\endcsname%
    \global\SB@creg{\\}%
  }%
}
\newcommand\replay{\@ifnextchar[\SB@replay\SB@@replay}
\newcommand\SB@replay{}
\def\SB@replay[#1]{%
  \@ifundefined{SB@cr@#1}{\SB@errreg{#1}}{%
    \SB@trackchfalse%
    \global\expandafter\let\expandafter\SB@creg%
      \csname SB@cr@#1\endcsname%
    \global\SB@ctail\SB@creg%
  }%
}
\newcommand\SB@@replay{%
  \SB@trackchfalse%
  \global\SB@ctail\SB@creg%
}
\newcommand\SB@rechord{}
\newcommand\SB@@rechord{%
  \SB@ifempty\SB@ctail{%
    \SB@errreplay%
    \SB@toks{}%
    \let\SB@donext\@gobble%
  }{%
    \SB@lop\SB@ctail\SB@toks%
    \let\SB@donext\SB@chord%
    \let\SB@noreplay\@gobble%
  }%
  \expandafter\SB@donext\the\SB@toks]%
}
\newif\ifSB@nohat
\newcommand\SB@noreplay{}
\let\SB@noreplay\@firstofone
\newlength\SB@fretwidth
\setlength\SB@fretwidth{6\p@}
\newcommand\SB@fretnum[1]{{%
  \sffamily\fontsize\@xpt\@xpt\selectfont#1%
}}
\newcommand\SB@onfret[1]{%
  \rlap{\hbox to\SB@fretwidth{\hfil\vrule\@height6\p@\hfil}}%
  \hbox to\SB@fretwidth{\hfil#1\hfil}%
}
\newcommand\SB@atopfret[1]{%
  \hbox to\SB@fretwidth{\hfil#1\hfil}%
}
\newcommand\SB@fretbar{%
  \nointerlineskip%
  \hbox to\SB@dimen{%
    \advance\SB@dimen-\SB@fretwidth%
    \advance\SB@dimen.4\p@%
    \hfil%
    \vrule\@width\SB@dimen\@height.4\p@\@depth\z@%
    \hfil%
  }%
  \nointerlineskip%
}
\newcommand\SB@topempty{\SB@atopfret\relax}
\newcommand\SB@topX{\SB@atopfret{%
  \hbox{%
    \kern-.2\p@%
    \fontencoding{OMS}\fontfamily{cmsy}%
    \fontseries{m}\fontshape{n}%
    \fontsize\@viipt\@viipt\selectfont\char\tw@%
    \kern-.2\p@%
  }%
}}
\newcommand\SB@topO{\SB@atopfret{%
  \vrule\@width\z@\@height4.3333\p@\@depth.8333\p@%
  \lower.74\p@\hbox{%
    \fontencoding{OMS}\fontfamily{cmsy}%
    \fontseries{m}\fontshape{n}%
    \fontsize\@xpt\@xpt\selectfont\char14%
  }%
}}
\newcommand\SB@fretempty{\SB@onfret\relax}
\newcommand\SB@frethit{\SB@onfret{%
  \hbox{%
    \fontencoding{OMS}\fontfamily{cmsy}%
    \fontseries{m}\fontshape{n}%
    \fontsize\@xiipt\@xiipt\selectfont\char15%
  }%
}}
\newcommand\SB@finger[1]{%
  \SB@atopfret{\sffamily\fontsize\@vipt\@vipt\selectfont#1}%
}
\newif\ifSB@gettabind\SB@gettabindfalse
\SB@newdimen\SB@tabindent
\newcommand\SB@targfret{}
\newcommand\SB@targstr{}
\newcommand\SB@targfing{}
\newcommand\SB@csify[2]{%
  \SB@toks{}%
  \SB@cnt\z@%
  \SB@@csify#2\SB@@csify%
  \edef#1{\the\SB@toks}%
}
\newcommand\SB@@csify[1]{%
  \ifx#1\SB@@csify\else%
    \advance\SB@cnt\@ne%
    \SB@toks\expandafter{\the\SB@toks\csname#1\endcsname}%
    \expandafter\SB@@csify%
  \fi%
}
\newcommand\SB@gttop{%
  \let\X\SB@topX\let\0\SB@topO\let\O\0\let\1\SB@topempty%
  \let\2\1\let\3\1\let\4\1\let\5\1%
  \let\6\1\let\7\1\let\8\1\let\9\1%
}
\newcommand\SB@gtinit{%
  \let\X\SB@fretempty\let\0\X\let\O\X\let\1\SB@frethit%
  \let\2\X\let\3\X\let\4\X\let\5\X%
  \let\6\X\let\7\X\let\8\X\let\9\X%
}
\newcommand\SB@gtinc{%
  \let\9\8\let\8\7\let\7\6\let\6\5\let\5\4%
  \let\4\3\let\3\2\let\2\1\let\1\0%
}
\newcommand\SB@gtset[2]{%
  \let\X#1\let\0\X\let\O\X%
  \def\1{#21}\def\2{#22}\def\3{#23}%
  \def\4{#24}\def\5{#25}\def\6{#26}%
  \def\7{#27}\def\8{#28}\def\9{#29}%
}
\newcommand\SB@gtmax[1]{\ifnum\SB@cnt<#1\SB@cnt#1\fi}
\newcommand\gtab{\SB@begincname\SB@gtab}
\newcommand*\SB@gtab[1]{%
  \SB@endcname%
  \begingroup%
    \catcode`:12\relax%
    \SB@@gtab{#1}%
}
\newcommand*\SB@@gtab[2]{%
  \endgroup%
  \ifnum\SB@transposefactor=\z@%
    \SB@@@gtab{#1}{#2}%
  \else%
    \begingroup%
      \let\gtab\SB@@@gtab%
      \gtabtrans{#1}{#2}%
    \endgroup%
  \fi%
}
\newcommand\gtabtrans[2]{\transposehere{#1}}
\newcommand\SB@@@gtab[2]{%
  \let\SB@targfret\@empty%
  \let\SB@targstr\@empty%
  \let\SB@targfing\@empty%
  \SB@tabargs#2:::\SB@tabargs%
  \ifx\SB@targstr\@empty%
    \def\SB@targstr{\0\0\0\0\0\0}%
  \fi%
  \ifvmode\leavevmode\fi%
  \vbox{%
    \normalfont\normalsize%
    \setbox\SB@box\hbox{%
      \thinspace{\printchord{\transposehere{#1}\strut}}\thinspace%
    }%
    \setbox\SB@boxii\hbox{\SB@fretnum{\SB@targfret}}%
    \setbox\SB@boxiii\hbox{{\SB@gttop\SB@targstr}}%
    \hsize\wd\SB@box%
    \ifSB@gettabind%
      \global\SB@tabindent\wd\SB@boxii%
      \global\advance\SB@tabindent.5\SB@fretwidth%
      \global\advance\SB@tabindent-.5\p@%
    \fi%
    \SB@dimen\wd\SB@boxii%
    \advance\SB@dimen\wd\SB@boxiii%
    \ifdim\hsize<\SB@dimen%
      \hsize\SB@dimen%
    \else\ifSB@gettabind%
      \SB@dimenii\hsize%
      \advance\SB@dimenii-\SB@dimen%
      \divide\SB@dimenii\tw@%
      \global\advance\SB@tabindent\SB@dimenii%
    \fi\fi%
    \hbox to\hsize{\hfil\unhbox\SB@box\hfil}%
    \kern-\p@\nointerlineskip%
    \hbox to\hsize{%
      \hfil%
      \vtop{\kern\p@\kern2\p@\box\SB@boxii}%
      \vtop{%
        \SB@dimen\wd\SB@boxiii%
        \box\SB@boxiii%
        \SB@cnt\minfrets%
        \SB@gtset\relax\SB@gtmax\SB@targstr%
        \SB@gtinit%
        \loop%
          \SB@fretbar\hbox{\SB@targstr}%
          \advance\SB@cnt\m@ne%
        \ifnum\SB@cnt>\z@\SB@gtinc\repeat%
        \SB@fretbar%
        \ifx\SB@targsfing\@empty\else%
          \kern1.5\p@%
          \SB@gtset\SB@topempty\SB@finger%
          \hbox{\SB@targfing}%
        \fi%
      }%
      \hfil%
    }%
    \kern3\p@%
  }%
  \SB@gettabindfalse%
}
\newcommand\SB@ctoken{} \def\SB@ctoken{:}
\newcommand\SB@tabargs{}
\def\SB@tabargs#1:#2:#3:#4\SB@tabargs{%
  \def\SB@temp{#4}%
  \ifx\SB@temp\@empty%
    \SB@csify\SB@targstr{#1}%
  \else\ifx\SB@temp\SB@ctoken%
    \SB@csify\SB@targstr{#1}%
    \ifnum\SB@cnt>\@ne%
      \SB@cntii\SB@cnt%
      \SB@csify\SB@targfing{#2}%
      \SB@cnt\SB@cntii%
    \else%
      \def\SB@targfret{#1}%
      \SB@csify\SB@targstr{#2}%
    \fi%
  \else%
    \def\SB@targfret{#1}%
    \SB@csify\SB@targfing{#3}%
    \SB@csify\SB@targstr{#2}%
  \fi\fi%
}
\newcommand\songchapter{%
  \let\SB@temp\@seccntformat%
  \def\@seccntformat##1{}%
  \@startsection{chapter}{0}{\z@}%
    {3.5ex\@plus1ex\@minus.2ex}%
    {.4ex\let\@seccntformat\SB@temp}%
    {\sffamily\bfseries\LARGE\centering}%
}
\newcommand\songsection{%
  \let\SB@temp\@seccntformat%
  \def\@seccntformat##1{}%
  \@startsection{section}{1}{\z@}%
    {3.5ex\@plus1ex\@minus.2ex}%
    {.4ex\let\@seccntformat\SB@temp}%
    {\sffamily\bfseries\LARGE\centering}%
}
\newenvironment{songs}[1]{%
  \ifSB@songsenv\SB@errnse\fi%
  \gdef\SB@indexlist{#1}%
  \SB@chkidxlst%
  \stepcounter{SB@songsnum}%
  \setcounter{songnum}{1}%
  \let\SB@sgroup\@empty%
  \ifinner\else\ifdim\pagetotal>\z@%
    \null\nointerlineskip%
  \fi\fi%
  \songcolumns\SB@numcols%
  \SB@songsenvtrue%
}{%
  \commitsongs%
  \global\let\SB@indexlist\@empty%
  \ifinner\else\clearpage\fi%
  \SB@songsenvfalse%
}
\newcounter{SB@songsnum}
\newcommand\songtarget[2]{%
  \ifnum\@ne=0\ifSB@pdf\ifx\pdfbookmark\undefined\else%
                       \ifx\pdfbookmark\relax\else1\fi\fi\fi\relax%
    \pdfbookmark[#1]{\thesongnum. \songtitle}{#2}%
  \else\ifx\hypertarget\undefined%
  \else\ifx\hypertarget\relax\else%
    \hypertarget{#2}{\relax}%
  \fi\fi\fi%
}
\newcommand\songlink{%
  \ifnum\@ne=0\ifx\hyperlink\undefined\else%
              \ifx\hyperlink\relax\else1\fi\fi\relax%
    \expandafter\hyperlink%
  \else%
    \expandafter\@gobble%
  \fi%
}
\newcommand\SB@indexlist{}
\newcommand\SB@allindexes{}
\let\SB@allindexes\@empty
\newcommand\SB@out{}
\let\SB@out\relax
\newcommand\SB@newindex[4]{%
  \expandafter\newcommand\csname SB@idxfilename@#3\endcsname{#4}%
  \expandafter\newcommand\csname SB@idxsel@#3\endcsname[3]{###1}%
  \expandafter\newcommand\csname SB@idxref@#3\endcsname{\thesongnum}%
  \xdef\SB@allindexes{%
    \ifx\SB@allindexes\@empty\else\SB@allindexes,\fi#3%
  }%
  \if@filesw%
    \ifx\SB@out\relax%
      \SB@newwrite\SB@out%
      \immediate\openout\SB@out=\jobname.sxc\relax%
    \fi%
    \immediate\write\SB@out{\noexpand\SB@iwrite{#3}{#2}}%
  \fi%
}
\newcommand\newindex{\SB@newindex1{TITLE INDEX DATA FILE}}
\@onlypreamble\newindex
\newcommand\newscripindex{\SB@newindex2{SCRIPTURE INDEX DATA FILE}}
\@onlypreamble\newscripindex
\newcommand\newauthorindex{\SB@newindex3{AUTHOR INDEX DATA FILE}}
\@onlypreamble\newauthorindex
\newcommand\SB@cwrite[2]{%
  \ifx\SB@out\relax\else%
    \protected@write\SB@out\SB@keepactive{\protect\SB@iwrite{#1}{#2}}%
  \fi%
}
\newcommand\SB@keepactive{}
{\catcode`\~\active
 \catcode`\.12
 \def\\#1#2{%
   \endgroup
   \SB@app\gdef\SB@keepactive{\def#1{#2}}%
 }
 \def\SB@temp#1#2{%
   \SB@cnt#1\relax
   \loop
     \begingroup
       \uccode`\~\SB@cnt
       \uccode`\.\SB@cnt
     \uppercase{\\~.}
   \ifnum\SB@cnt<#2\relax
     \advance\SB@cnt\@ne
   \repeat
 }
 \SB@temp{1}{8}
 \SB@temp{11}{11}
 \SB@temp{14}{91}
 \SB@temp{93}{255}
}
\newcommand\SB@iwrite[2]{%
  \def\SB@tempii{#1}%
  \ifx\SB@temp\SB@tempii%
    \SB@toks{#2}%
    \immediate\write\SB@out{\the\SB@toks}%
  \fi%
}
\newcommand\SB@uncombine{%
  \ifx\SB@out\relax\else%
    \immediate\closeout\SB@out%
    \ifsongindexes%
      \@for\SB@temp:=\SB@allindexes\do{%
        \immediate\openout\SB@out=%
          \csname SB@idxfilename@\SB@temp\endcsname.sxd\relax%
        \begingroup\makeatletter\catcode`\%12\relax%
                   \input{\jobname.sxc}\endgroup%
        \immediate\closeout\SB@out%
      }%
    \fi%
  \fi%
}
\AtEndDocument{\SB@uncombine}
\SB@newbox\SB@songwrites
\newcommand\SB@addtoindex[2]{%
  \protected@edef\SB@tempii{#2}%
  \ifx\SB@tempii\@empty\else%
    \global\setbox\SB@songwrites\vbox{%
      \unvbox\SB@songwrites%
      \SB@cwrite{#1}{#2}%
      \SB@cwrite{#1}{\csname SB@idxref@#1\endcsname}%
      \SB@cwrite{#1}{song\theSB@songsnum-\thesongnum.%
                     \ifnum\c@section=\z@1\else2\fi}%
    }%
  \fi%
}
\newcommand\SB@addtoindexes[3]{%
  \@for\SB@temp:=\SB@indexlist\do{%
    \SB@addtoindex\SB@temp%
      {\csname SB@idxsel@\SB@temp\endcsname{#1}{#2}{#3}}%
  }%
}
\newcommand\SB@addtotitles[1]{%
  \@for\SB@temp:=\SB@indexlist\do{%
    \csname SB@idxsel@\SB@temp\endcsname%
      {\SB@addtoindex\SB@temp{#1}}{}{}%
  }%
}
\newcommand\SB@chkidxlst{%
  \let\SB@temp\SB@indexlist%
  \let\SB@indexlist\@empty%
  \@for\SB@tempii:=\SB@temp\do{%
    \@ifundefined{SB@idxsel@\SB@tempii}{\SB@errnoidx\SB@tempii}{%
      \ifx\SB@indexlist\@empty%
        \SB@toks\expandafter{\SB@tempii}%
      \else%
        \SB@toks\expandafter\expandafter\expandafter{%
          \expandafter\SB@indexlist\expandafter,\SB@tempii}%
      \fi%
      \edef\SB@indexlist{\the\SB@toks}%
    }%
  }%
}
\newcommand\indexentry{\@ifnextchar[{\SB@idxentry*}{\SB@@idxentry*}}
\newcommand\SB@idxentry{}
\def\SB@idxentry#1[#2]#3{{%
  \def\SB@indexlist{#2}%
  \SB@chkidxlst%
  \SB@addtoindexes{#1#3}{#3}{#3}%
}}
\newcommand\SB@@idxentry[2]{\SB@addtotitles{#1#2}}
\newcommand\indextitleentry{%
  \@ifnextchar[{\SB@idxentry{}}{\SB@@idxentry{}}%
}
\newcommand\indexsongsas[1]{%
  \@ifundefined{SB@idxref@#1}%
    {\SB@errnoidx{#1}\@gobble}%
    {\expandafter\renewcommand\csname SB@idxref@#1\endcsname}%
}
\newcommand\SB@idxcmd[3]{%
  \ifx\SB@allindexes\@empty%
    \SB@warnnoidx%
  \else\ifx\SB@out\relax\else%
    \@for\SB@temp:=\SB@allindexes\do{%
      \csname SB@idxsel@\SB@temp\endcsname%
        {\SB@@idxcmd{#1}}{\SB@@idxcmd{#2}}{\SB@@idxcmd{#3}}%
    }%
  \fi\fi%
}
\newcommand\SB@@idxcmd[1]{%
  \def\SB@tempii{#1}%
  \ifx\SB@tempii\@empty\else%
    \immediate\write\SB@out{%
      \noexpand\SB@iwrite{\SB@temp}{#1}%
    }%
  \fi%
}
\newcommand\authsepword[1]{}
\newcommand\authbyword[1]{}
\newcommand\authignoreword[1]{}
\newcommand\titleprefixword[1]{}
{\catcode`\%=12
 \gdef\authsepword#1{\SB@idxcmd{}{}{%sep #1}}
 \gdef\authbyword#1{\SB@idxcmd{}{}{%after #1}}
 \gdef\authignoreword#1{\SB@idxcmd{}{}{%ignore #1}}
 \gdef\titleprefixword#1{\SB@idxcmd{%prefix #1}{}{}}}
\@onlypreamble\authsepword
\@onlypreamble\authbyword
\@onlypreamble\authignoreword
\@onlypreamble\titleprefixword
\newcommand\SB@idxlineskip[1]{%
  \vskip#1\p@\@plus#1\p@\@minus#1\p@%
}
\newcommand\SB@ellipspread[2]{%
  \begingroup%
    \SB@dimen\z@%
    \def\SB@temp{#1}%
    \SB@toks{#2}%
    \setbox\SB@box\hbox{{%
      \SB@temp%
      \leaders\hbox to.5em{\hss.\hss}\hskip2em\@plus1fil%
      {\the\SB@toks}%
    }}%
    \ifdim\wd\SB@box>\hsize%
      \SB@balancerows%
    \else%
      \hbox to\hsize{\unhbox\SB@box}\par%
    \fi%
  \endgroup%
}
\newcommand\SB@balancerows{%
  \edef\SB@tempii{\the\SB@toks}%
  \setbox\SB@box\vbox{%
    \SB@toks\expandafter{\expandafter\\\the\SB@toks\\}%
    \SB@lop\SB@toks\SB@toks%
    \settowidth\SB@dimen{\the\SB@toks}%
    \advance\SB@dimen-.5em%
    \leftskip.5cm%
    {\hbadness\@M\hfuzz\maxdimen%
     \hskip-.5cm\relax\SB@temp\unskip\nobreak%
     \hskip\SB@dimen\nobreak%
     \rightskip2em\@plus1fil\par}%
    \setbox\SB@box\lastbox%
    \setbox\SB@box\hbox{%
      \unhbox\SB@box%
      \unskip\unskip\unpenalty%
      \unpenalty\unskip\unpenalty%
    }%
    \expandafter%
  }%
  \expandafter\SB@dimenii\the\wd\SB@box\relax%
  \SB@dimen\hsize%
  \advance\SB@dimen-.5cm%
  \setbox\SB@box\vbox{%
    \SB@multiline{\hbadness\@M\hfuzz\maxdimen}%
  }%
  \SB@dimeniii.5\SB@dimen%
  \SB@dimeniv\SB@dimeniii%
  \loop%
    \SB@dimeniv.5\SB@dimeniv%
    \setbox\SB@boxii\vbox{%
      \SB@dimen\SB@dimeniii%
      \SB@multiline{\hbadness\@M\hfuzz\maxdimen}%
    }%
    \ifnum\SB@cnt<\@M%
      \ifdim\ht\SB@boxii>\ht\SB@box%
        \advance\SB@dimeniii\SB@dimeniv%
      \else%
        \SB@dimen\SB@dimeniii%
        \advance\SB@dimeniii-\SB@dimeniv%
      \fi%
    \else%
      \advance\SB@dimeniii\SB@dimeniv%
    \fi%
  \ifdim\SB@dimeniv>2\p@\repeat%
  \setbox\SB@box\box\voidb@x%
  \setbox\SB@boxii\box\voidb@x%
  \SB@multiline\relax%
}
\newcommand\SB@multiline[1]{%
  \begingroup%
    \SB@dimen-\SB@dimen%
    \advance\SB@dimen\hsize%
    \SB@dimenii-\SB@dimenii%
    \advance\SB@dimenii\SB@dimen%
    {#1\hskip-.5cm\relax\SB@temp\unskip\nobreak%
     \SB@maxmin\SB@dimenii<{1.5em}%
     \leftskip.5cm\rightskip2em\@plus1fil%
     \interlinepenalty\@M%
     \leaders\hbox to.5em{\hss.\hss}\hskip\SB@dimenii\@plus1fill%
     \nobreak{\SB@tempii\kern-2em}%
     \par\global\SB@cnt\badness}%
  \endgroup%
}%
\newcommand\SB@idxitemsep{%
  ,\kern-2em\penalty-8\hskip2.33em\@minus.11em%
  \hskip-\SB@dimen\@plus-1fill%
  \vadjust{}\nobreak%
  \hskip\SB@dimen\@plus1fill\relax%
}
\newenvironment{idxblock}[1]{}{}
\newcommand\idxentry[2]{}
\newcommand\idxaltentry[2]{}
\newenvironment{SB@lgidx}[1]{}{}
\newenvironment{SB@smidx}[1]{}{}
\newcommand\SB@idxsetup{%
  \hsize\SB@colwidth%
  \parskip\z@skip\parfillskip\z@skip\parindent\z@%
  \baselineskip\f@size\p@\@plus\p@\@minus\p@%
  \lineskiplimit\z@\lineskip\p@\@plus\p@\@minus\p@%
  \hyphenpenalty\@M\exhyphenpenalty\@M%
}
\newcommand\SB@makeidxcolumn{%
  \ifdim\ht\SB@box=\z@%
    \hskip\hsize\relax%
  \else%
    \splittopskip\z@skip\splitmaxdepth\maxdepth%
    \vsplit\SB@box to\SB@dimen%
    \global\setbox\SB@box\vbox{%
      \SB@idxsetup%
      \splitbotmark%
      \unvbox\SB@box%
    }%
  \fi%
}
\newcommand\SB@oneidxpage{}
\newcommand\SB@displayindex[3]{%
  \ifsongindexes\begingroup%
    \SB@colwidth\hsize%
    \advance\SB@colwidth-#1\columnsep%
    \advance\SB@colwidth\columnsep%
    \divide\SB@colwidth#1%
    \setbox\SB@envbox\vbox{%
      \let\SB@temp\songsection%
      \ifx\chapter\undefined\else%
        \ifx\chapter\relax\else%
          \let\SB@temp\songchapter%
        \fi%
      \fi%
      \SB@temp{#2}%
    }%
    \IfFileExists{\csname SB@idxfilename@#3\endcsname.sbx}{%
      \ifsepindexes%
        \global\setbox\SB@box\vbox{%
          \null%
          \vfil%
          \unvcopy\SB@envbox%
          \vskip.5in\@minus.3in\relax%
          \hbox to\hsize{%
            \hfil%
            \vbox{%
              \SB@idxsetup%
              \renewenvironment{idxblock}[1]%
                {\begin{SB@smidx}{####1}}{\end{SB@smidx}}%
              \let\\\SB@idxitemsep%
              \input{\csname SB@idxfilename@#3\endcsname.sbx}%
            }%
            \hfil%
          }%
          \vskip\z@\@plus2fil\relax%
        }%
        {\vbadness\@M\vfuzz\maxdimen%
         \splitmaxdepth\maxdepth\splittopskip\z@skip%
         \global\setbox\SB@boxii\vsplit\SB@box to\textheight}%
        \ifvoid\SB@box%
          \box\SB@boxii%
        \else%
          \SB@lgindex{#1}{#3}%
        \fi%
      \else%
        \SB@lgindex{#1}{#3}%
      \fi%
    }%
    {%
      \ifsepindexes%
        \vbox to\textheight{%
          \vfil%
          \unvbox\SB@envbox%
          \vskip1em\relax%
          \hbox to\hsize{\hfil[Index not yet generated.]\hfil}%
          \vskip\z@\@plus2fil\relax%
        }%
      \else%
        \unvbox\SB@envbox%
        \hbox to\hsize{\hfil[Index not yet generated.]\hfil}%
      \fi%
    }%
    \ifsepindexes\clearpage\fi%
  \endgroup\fi%
}
\newcommand\SB@lgindex[2]{%
  \global\setbox\SB@box\vbox{%
    \renewenvironment{idxblock}[1]%
      {\begin{SB@lgidx}{##1}}{\end{SB@lgidx}}%
    \let\\\SB@idxitemsep%
    \SB@idxsetup%
    \input{\csname SB@idxfilename@#2\endcsname.sbx}%
    \unskip%
  }%
  \SB@toks{\SB@makeidxcolumn}%
  \SB@cnt#1\relax%
  \loop\ifnum\SB@cnt>\@ne%
    \SB@toks\expandafter{\the\SB@toks%
      \kern\columnsep\SB@makeidxcolumn}%
    \advance\SB@cnt\m@ne%
  \repeat%
  \edef\SB@oneidxpage{\the\SB@toks}%
  \unvbox\SB@envbox%
  \vskip.2in\relax%
  \nointerlineskip%
  \null%
  \nointerlineskip%
  \SB@cnt\vbadness\vbadness\@M%
  \SB@dimenii\vfuzz\vfuzz\maxdimen%
  \loop%
    \SB@dimen\textheight%
    \ifinner\else\kern\z@\advance\SB@dimen-\pagetotal\fi%
    \global\setbox\SB@boxii\copy\SB@box%
    \global\setbox\SB@boxiii\hbox{\SB@oneidxpage}%
    \ifdim\ht\SB@box>\z@%
      \box\SB@boxiii%
      \vfil\break%
  \repeat%
  \SB@dimenii\ht\SB@boxii%
  \divide\SB@dimenii#1\relax%
  \SB@maxmin\SB@dimen>\SB@dimenii%
  \loop%
    \global\setbox\SB@box\copy\SB@boxii%
    \global\setbox\SB@boxiii\hbox{\SB@oneidxpage}%
    \ifdim\ht\SB@box>\z@%
      \advance\SB@dimen\p@%
  \repeat%
  \box\SB@boxiii%
  \global\setbox\SB@boxii\box\voidb@x%
  \vbadness\SB@cnt\vfuzz\SB@dimenii%
}
\newcommand\showindex[3][0]{%
  \@ifundefined{SB@idxsel@#3}{\SB@errnoidx{#3}}{%
    \expandafter\let\expandafter\SB@temp\csname SB@idxsel@#3\endcsname%
    \SB@cnt#1\relax%
    \ifnum\SB@cnt<\@ne\SB@cnt\SB@temp232\relax\fi%
    \expandafter\SB@temp%
    \expandafter\SB@maketitleindex%
    \expandafter\SB@makescripindex%
    \expandafter\SB@makeauthorindex%
    \expandafter{\the\SB@cnt}%
    {#2}{#3}%
  }%
}
\newcommand\SB@maketitleindex{%
  \ifnum\idxheadwidth>\z@%
    \renewenvironment{SB@lgidx}[1]{
      \hbox{\SB@colorbox\idxbgcolor{\vbox{%
        \hbox to\idxheadwidth{{\idxheadfont\relax##1}\hfil}%
      }}}%
      \nobreak\vskip3\p@\@plus2\p@\@minus2\p@\nointerlineskip%
    }{\penalty-50\vskip5\p@\@plus5\p@\@minus4\p@}%
  \else%
    \renewenvironment{SB@lgidx}[1]{}{}%
  \fi%
  \renewenvironment{SB@smidx}[1]{}{}%
  \renewcommand\idxentry[2]{%
    \SB@ellipspread{\idxtitlefont\relax\ignorespaces##1\unskip}%
                   {{\idxrefsfont\relax##2}}%
  }%
  \renewcommand\idxaltentry[2]{%
    \SB@ellipspread{\idxlyricfont\relax\ignorespaces##1\unskip}%
                   {{\idxrefsfont\relax##2}}%
  }%
  \SB@displayindex%
}
\newcommand\SB@idxcolhead{}
\newcommand\SB@idxheadsep{{%
  \SB@dimen4\p@%
  \advance\SB@dimen-\prevdepth%
  \SB@maxmin\SB@dimen<\z@%
  \SB@dimenii\SB@dimen%
  \SB@maxmin\SB@dimenii>\p@%
  \vskip\SB@dimen\@plus\p@\@minus\SB@dimenii%
}}
\newcommand\SB@idxcont[1]{%
  \hbox to\hsize{{\idxcont{#1}}\hfil}%
  \nobreak%
  \SB@idxheadsep\nointerlineskip%
}
\newcommand\SB@makescripindex{%
  \renewenvironment{SB@lgidx}[1]{%
    \gdef\SB@idxcolhead{##1}%
    \hbox to\hsize{{\idxbook{##1}}\hfil}%
    \nobreak%
    \SB@idxheadsep\nointerlineskip%
  }{%
    \mark{\noexpand\relax}%
    \penalty-20\vskip3\p@\@plus3\p@\relax%
  }%
  \renewenvironment{SB@smidx}[1]
    {\begin{SB@lgidx}{##1}}{\end{SB@lgidx}}%
  \renewcommand\idxentry[2]{%
    \SB@ellipspread{\hskip.25cm\idxscripfont\relax##1}%
                   {{\idxrefsfont\relax##2}}%
    \SB@toks\expandafter{\SB@idxcolhead}%
    \mark{\noexpand\SB@idxcont{\the\SB@toks}}%
  }%
  \renewcommand\idxaltentry[2]{\SB@erridx{a scripture}}%
  \SB@displayindex%
}
\newcommand\SB@makeauthorindex{%
  \renewenvironment{SB@lgidx}[1]{}{}%
  \renewenvironment{SB@smidx}[1]{}{}%
  \renewcommand\idxentry[2]{%
    \SB@ellipspread{{\idxauthfont\relax\sfcode`.\@m##1}}%
                   {{\idxrefsfont##2}}%
  }%
  \renewcommand\idxaltentry[2]{\SB@erridx{an author}}%
  \SB@displayindex%
}
\newcommand\SB@Error{\PackageError{songs}}
\newcommand\SB@Warn{\PackageWarning{songs}}
\newcommand\SB@errspos{%
  \SB@Error{Illegal \protect\songpos\space argument}{The argume%
  nt to \protect\songpos\space must be a number from 0 to 3.}%
}
\newcommand\SB@errnse{%
  \SB@Error{Nested songs environments are not supported}{End th%
  e previous songs environment before beginning the next one.}%
}
\newcommand\SB@errpl{%
  \SB@Error{\protect\includeonlysongs\space not permitted with%
  in a songs environment}{\protect\includeonlysongs\space can o%
  nly be used in the document preamble or between songs environ%
  ments in the document body.}%
}
\newcommand\SB@errrtopt{%
  \SB@Error{Cannot display chords in a rawtext dump}{You have u%
  sed the rawtext option in the \protect\usepackage\space lin%
  e and have either used the chorded option as well or have use%
  d the \protect\chordson\space macro subsequently.}%
}
\newcommand\SB@warnrc{%
  \SB@Warn{The \protect\repchoruses\space feature will not wor%
  k when the number of columns is set to zero}%
}
\newcommand\SB@warnnoidx{%
  \SB@Warn{Index command has no effect since no indexes are ye%
  t declared}%
}
\newcommand\SB@errboo{%
  \SB@Error{Encountered \protect\beginsong\space without seein%
  g an \protect\endsong\space for the previous song}%
  {Song \thesongnum\space might be missing a%
  n \protect\endsong\space line.}%
}
\newcommand\SB@errbor{%
  \SB@Error{Encountered \protect\beginsong\space without seein%
  g an \protect\endscripture\space for the preceding scriptur%
  e quotation}{A scripture quotation appearing after son%
  g \thesongnum\space might be missing a%
  n \protect\endscripture\space line.}%
}
\newcommand\SB@erreov{%
  \SB@Error{Encountered \protect\endsong\space without seein%
  g an \protect\endverse\space for the preceding verse}{Son%
  g \thesongnum\space has a \protect\beginverse\space%
  line with no matching \protect\endverse\space line.}%
}
\newcommand\SB@erreoc{%
  \SB@Error{Encountered \protect\endsong\space without seein%
  g an \protect\endchorus\space for the preceding chorus}{Son%
  g \thesongnum\space has a \protect\beginchorus\space%
  line with no matching \protect\endchorus\space line.}%
}
\newcommand\SB@erreor{%
  \SB@Error{Encountered \protect\endsong\space without seein%
  g an \protect\endscripture for the preceding scripture quot%
  e}{A scripture quote appearing before song \thesongnum\space%
  ended with \protect\endsong\space instead of wit%
  h \protect\endscripture.}%
}
\newcommand\SB@erreot{%
  \SB@Error{Encountered \protect\endsong\space with no matchin%
  g \protect\beginsong}{Before song \thesongnum\space there wa%
  s an \protect\endsong\space with no matchin%
  g \protect\beginsong.}%
}
\newcommand\SB@errbvv{%
  \SB@Error{Encountered \protect\beginverse\space without seein%
  g an \protect\endverse\space for the preceding verse}{Son%
  g \thesongnum\space might have a verse that has n%
  o \protect\endendverse\space line.}%
}
\newcommand\SB@errbvc{%
  \SB@Error{Encountered \protect\beginverse\space without seein%
  g an \protect\endchorus\space for the preceding chorus}{Son%
  g \thesongnum\space might have a chorus that has n%
  o \protect\endchorus\space line.}%
}
\newcommand\SB@errbvt{%
  \SB@Error{Encountered \protect\beginverse\space without firs%
  t seeing a \protect\beginsong\space line}{Before son%
  g \thesongnum, there is a \protect\beginverse\space line no%
  t contained in any song.}%
}
\newcommand\SB@errevc{%
  \SB@Error{Encountered \protect\endverse\space while process%
  ing a chorus}{Song \thesongnum\space might hav%
  e a \protect\beginchorus\space concluded by a%
  n \protect\endverse\space instead of an \protect\endchorus.}%
}
\newcommand\SB@errevo{%
  \SB@Error{Encountered \protect\endverse\space without firs%
  t seeing a \protect\beginverse}{Song \thesongnum\space m%
  ight have an \protect\endverse\space with no matchin%
  g \protect\beginverse.}%
}
\newcommand\SB@errevt{%
  \SB@Error{Encountered an \protect\endverse\space outside o%
  f any song}{Before song \thesongnum, there is a%
  n \protect\endverse\space line not preceded b%
  y a \protect\beginsong\space line.}%
}
\newcommand\SB@erretex{%
  \SB@Error{The \protect\repchoruses\space feature requires e-%
  TeX compatibility}{Your version of LaTeX2e does not appear t%
  o be e-TeX compatible. Find a distribution that includes e-T%
  eX support in order to use this feature.}%
}
\newcommand\SB@errbcv{%
  \SB@Error{Encountered \protect\beginchorus\space without see%
  ing an \protect\endverse\space for the preceding verse}{Son%
  g \thesongnum\space might hav%
  e a \protect\beginverse\space with no match%
  ing \protect\endverse.}%
}
\newcommand\SB@errbcc{%
  \SB@Error{Encountered \protect\beginchorus\space without see%
  ing an \protect\endchorus\space for the preceding chorus}%
  {Song \thesongnum\space might have a \protect\beginchorus%
  \space with no matching \protect\endchorus.}%
}
\newcommand\SB@errbct{%
  \SB@Error{Encountered \protect\beginchorus\space without see%
  ing a \protect\beginsong\space line first}{After son%
  g \thesongnum\space there is a \protect\beginchorus\space%
  line outside of any song.}%
}
\newcommand\SB@errecv{%
  \SB@Error{Encountered an \protect\endchorus\space while proc%
  essing a verse}{Song \thesongnum\space might hav%
  e a \protect\beginverse\space concluded by \protect\endchorus%
  \space instead of \protect\endverse.}%
}
\newcommand\SB@erreco{%
  \SB@Error{Encountered \protect\endchorus\space without firs%
  t seeing a \protect\beginchorus}{Song \thesongnum\space m%
  ight have an \protect\endchorus\space with no match%
  ing \protect\beginchorus.}%
}
\newcommand\SB@errect{%
  \SB@Error{Encountered an \protect\endchorus\space outside o%
  f any song}{Before song \thesongnum, there is a%
  n \protect\endchorus\space line not preceded b%
  y a \protect\beginsong\space line.}%
}
\newcommand\SB@errbro{%
  \SB@Error{Missing \protect\endsong}%
  {Nested song and intersong environments are not supported%
  . Song \thesongnum\space might be missing a%
  n \protect\endsong\space line.}%
}
\newcommand\SB@errbrr{%
  \SB@Error{Nested intersong environments are not supported}%
  {A scripture quote or other intersong environment before s%
  ong \thesongnum\space is missing its ending line.}%
}
\newcommand\SB@errero{%
  \SB@Error{Encountered an \protect\endscripture\space whil%
  e processing a song}{Song \thesongnum\space ends wit%
  h \protect\endscripture\space when it should end wit%
  h \protect\endsong.}%
}
\newcommand\SB@errert{%
  \SB@Error{Encountered an \protect\endscripture\space with%
  out first seeing a \protect\beginscripture}{Before son%
  g \thesongnum, there is an \protect\endscripture\space w%
  ith no matching \protect\beginscripture.}%
}
\newcommand\SB@errscrip[1]{%
  \SB@Error{Encountered a \protect#1\space outside a scriptu%
  re quote}{\protect#1\space can only appear betwee%
  n \protect\beginscripture\space an%
  d \protect\endscripture\space lines.}%
}
\newcommand\SB@errchord{%
  \SB@Error{Song \thesongnum\space seems to have chord%
  s that appear outside of any verse or chorus}{All chords a%
  nd lyrics should appear between \protect\beginverse\space%
  and \protect\endverse, or between \protect\beginchorus\space%
  and \protect\endchorus.}%
}
\newcommand\SB@errreplay{%
  \SB@Error{Replayed chord has no matching chord}{Son%
  g \thesongnum\space uses \protect^ more times than the%
  re are chords in the previously memorized verse.}%
}
\newcommand\SB@errreg[1]{%
  \SB@Error{Unknown chord-replay register name: #1}{Chord-re%
  play registers must be declared with \protect\newchords.}%
}
\newcommand\SB@errdup[1]{%
  \SB@Error{Duplicate definition of chord-replay register%
  : #1}{\protect\newchords\space was used to declare the sa%
  me chord-replay register twice.}%
}
\newcommand\SB@errmbar{%
  \SB@Error{Song \thesongnum\space seems to have measur%
  e bars that appear outside of any verse or chorus}{All mea%
  sure bars (produced with \protect\mbar\space or |) must ap%
  pear between \protect\beginverse\space an%
  d \protect\endverse, or between \protect\beginchorus\space%
  and \protect\endchorus.}%
}
\newcommand\SB@errtab{%
  \SB@Error{Invalid argument to \protect\gtab\space macro. R%
   eplacing it with \protect\0.}{Valid arguments consist onl%
   y of: X, O, 0, 1, 2, 3, or 4.}%
}
\newcommand\SB@errnoidx[1]{%
  \SB@Error{Unknown index identifier: #1}{This index identifie%
  r was never declared using \protect\newindex.}%
}
\newcommand\SB@erridx[1]{%
  \SB@Error{\protect\idxaltentry\space not allowed in #1 index}%
  {This error should not occur. The index generation routines ha%
  ve malfunctioned. Try deleting all temporary files and then re%
  compiling.}%
}
\newif\ifchorded
\newif\iflyric\lyrictrue
\newif\ifslides
\newif\ifmeasures
\newif\ifpartiallist
\newif\ifrepchorus
\newif\iftranscapos
\newif\ifnolyrics
\newif\ifrawtext
\newif\ifsongindexes\songindexestrue
\newif\ifsepindexes\sepindexestrue
\newif\ifpagepreludes
\newif\ifSB@colorboxes
\IfFileExists{color.sty}\SB@colorboxestrue\SB@colorboxesfalse
\newif\ifSB@omitscrip
\newcommand\nolyrics{}
\let\nolyrics\nolyricstrue
\newcommand\pagepreludes{\pagepreludestrue\songpos0}
\SB@chordson
\ProcessOptions\relax
\ifSB@colorboxes
  \RequirePackage{color}
  \definecolor{SongbookShade}{gray}{.80}
  \newcommand\SB@colorbox[2]{%
    \ifx\@empty#1%
      \vbox{%
        \kern3\p@%
        \hbox{\kern3\p@{#2}\kern3\p@}%
        \kern3\p@%
      }%
    \else%
      \colorbox{#1}{#2}%
    \fi%
  }
\else
  \newcommand\SB@colorbox[2]{\vbox{%
    \kern3\p@%
    \hbox{\kern3\p@{#2}\kern3\p@}%
    \kern3\p@%
  }}
\fi
\ifrawtext
  \SB@newwrite\SB@txtout
  \immediate\openout\SB@txtout=\jobname.txt
  \newif\ifSB@doEOL
  {\catcode`\^^M12 %
   \catcode`\^^J12 %
   \gdef\SB@printEOL{\ifSB@doEOL^^M^^J\fi}}
  {\catcode`#12\gdef\SB@hash{#}}
  {\catcode`&12\gdef\SB@amp{&}}
  \renewcommand\SB@@@beginsong{%
    \begingroup%
      \def\'{}\def\`{}\def\v{}\def\u{}\def\={}\def\^{}%
      \def\.{}\def\H{}\def\~{}\def\"{}\def\t{}%
      \def\copyright{(c)}%
      \let~\space%
      \let\par\SB@printEOL%
      \let\#\SB@hash%
      \let\&\SB@amp%
      \catcode`|9 %
      \catcode`*9 %
      \catcode`^9 %
      \def\[##1]{}%
      \resettitles%
      \immediate\write\SB@txtout{\thesongnum. \songtitle}%
      \nexttitle%
      \foreachtitle{\immediate\write\SB@txtout{(\songtitle)}}%
      \ifx\songauthors\@empty\else%
         \immediate\write\SB@txtout{\songauthors}%
      \fi%
      \ifx\SB@rawrefs\@empty\else%
         \immediate\write\SB@txtout{\SB@rawrefs}%
      \fi%
      \immediate\write\SB@txtout{}%
      \SB@doEOLfalse%
      \obeylines%
  }
  \renewcommand\SB@endsong{%
      \SB@doEOLtrue%
      \immediate\write\SB@txtout{\songcopyright\space%
        \songlicense\SB@printEOL}%
    \endgroup%
    \SB@insongfalse%
    \stepcounter{songnum}%
  }
  \def\SB@parsesrefs#1{\def\songrefs{#1}}
  \long\def\beginverse#1#2\endverse{%
    \SB@doEOLtrue\begingroup%
      \def\textnote##1{##1}%
      \def\SB@temp{#1}%
      \def\SB@star{*}%
      \ifx\SB@temp\SB@star%
        \immediate\write\SB@txtout{\@gobble#2}%
      \else%
        \immediate\write\SB@txtout{#2}%
      \fi%
    \endgroup\SB@doEOLfalse}
  \long\def\beginchorus#1\endchorus{%
    \SB@doEOLtrue\begingroup%
      \def\textnote##1{##1}%
      \immediate\write\SB@txtout{Chorus:#1}%
    \endgroup\SB@doEOLfalse}
  \long\def\beginscripture#1\endscripture{}
  \def\musicnote#1{}
  \def\textnote#1{%
    \SB@doEOLtrue%
    \immediate\write\SB@txtout{#1\SB@printEOL}%
    \SB@doEOLfalse}
  \def\brk{}
  \def\rep#1{(x#1)}
  \def\echo#1{(#1)}
  \def\mbar#1#2{}
  \def\lrep{}
  \def\rrep{}
  \def\nolyrics{}
  \renewcommand\memorize[1][]{}
  \renewcommand\replay[1][]{}
\fi
\endinput
%%
%% End of file `songs.sty'.

\end{songs}

\end{document}

